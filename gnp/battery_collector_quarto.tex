% Options for packages loaded elsewhere
% Options for packages loaded elsewhere
\PassOptionsToPackage{unicode}{hyperref}
\PassOptionsToPackage{hyphens}{url}
\PassOptionsToPackage{dvipsnames,svgnames,x11names}{xcolor}
%
\documentclass[
  letterpaper,
  DIV=11,
  numbers=noendperiod]{scrartcl}
\usepackage{xcolor}
\usepackage{amsmath,amssymb}
\setcounter{secnumdepth}{5}
\usepackage{iftex}
\ifPDFTeX
  \usepackage[T1]{fontenc}
  \usepackage[utf8]{inputenc}
  \usepackage{textcomp} % provide euro and other symbols
\else % if luatex or xetex
  \usepackage{unicode-math} % this also loads fontspec
  \defaultfontfeatures{Scale=MatchLowercase}
  \defaultfontfeatures[\rmfamily]{Ligatures=TeX,Scale=1}
\fi
\usepackage{lmodern}
\ifPDFTeX\else
  % xetex/luatex font selection
\fi
% Use upquote if available, for straight quotes in verbatim environments
\IfFileExists{upquote.sty}{\usepackage{upquote}}{}
\IfFileExists{microtype.sty}{% use microtype if available
  \usepackage[]{microtype}
  \UseMicrotypeSet[protrusion]{basicmath} % disable protrusion for tt fonts
}{}
\makeatletter
\@ifundefined{KOMAClassName}{% if non-KOMA class
  \IfFileExists{parskip.sty}{%
    \usepackage{parskip}
  }{% else
    \setlength{\parindent}{0pt}
    \setlength{\parskip}{6pt plus 2pt minus 1pt}}
}{% if KOMA class
  \KOMAoptions{parskip=half}}
\makeatother
% Make \paragraph and \subparagraph free-standing
\makeatletter
\ifx\paragraph\undefined\else
  \let\oldparagraph\paragraph
  \renewcommand{\paragraph}{
    \@ifstar
      \xxxParagraphStar
      \xxxParagraphNoStar
  }
  \newcommand{\xxxParagraphStar}[1]{\oldparagraph*{#1}\mbox{}}
  \newcommand{\xxxParagraphNoStar}[1]{\oldparagraph{#1}\mbox{}}
\fi
\ifx\subparagraph\undefined\else
  \let\oldsubparagraph\subparagraph
  \renewcommand{\subparagraph}{
    \@ifstar
      \xxxSubParagraphStar
      \xxxSubParagraphNoStar
  }
  \newcommand{\xxxSubParagraphStar}[1]{\oldsubparagraph*{#1}\mbox{}}
  \newcommand{\xxxSubParagraphNoStar}[1]{\oldsubparagraph{#1}\mbox{}}
\fi
\makeatother

\usepackage{color}
\usepackage{fancyvrb}
\newcommand{\VerbBar}{|}
\newcommand{\VERB}{\Verb[commandchars=\\\{\}]}
\DefineVerbatimEnvironment{Highlighting}{Verbatim}{commandchars=\\\{\}}
% Add ',fontsize=\small' for more characters per line
\usepackage{framed}
\definecolor{shadecolor}{RGB}{241,243,245}
\newenvironment{Shaded}{\begin{snugshade}}{\end{snugshade}}
\newcommand{\AlertTok}[1]{\textcolor[rgb]{0.68,0.00,0.00}{#1}}
\newcommand{\AnnotationTok}[1]{\textcolor[rgb]{0.37,0.37,0.37}{#1}}
\newcommand{\AttributeTok}[1]{\textcolor[rgb]{0.40,0.45,0.13}{#1}}
\newcommand{\BaseNTok}[1]{\textcolor[rgb]{0.68,0.00,0.00}{#1}}
\newcommand{\BuiltInTok}[1]{\textcolor[rgb]{0.00,0.23,0.31}{#1}}
\newcommand{\CharTok}[1]{\textcolor[rgb]{0.13,0.47,0.30}{#1}}
\newcommand{\CommentTok}[1]{\textcolor[rgb]{0.37,0.37,0.37}{#1}}
\newcommand{\CommentVarTok}[1]{\textcolor[rgb]{0.37,0.37,0.37}{\textit{#1}}}
\newcommand{\ConstantTok}[1]{\textcolor[rgb]{0.56,0.35,0.01}{#1}}
\newcommand{\ControlFlowTok}[1]{\textcolor[rgb]{0.00,0.23,0.31}{\textbf{#1}}}
\newcommand{\DataTypeTok}[1]{\textcolor[rgb]{0.68,0.00,0.00}{#1}}
\newcommand{\DecValTok}[1]{\textcolor[rgb]{0.68,0.00,0.00}{#1}}
\newcommand{\DocumentationTok}[1]{\textcolor[rgb]{0.37,0.37,0.37}{\textit{#1}}}
\newcommand{\ErrorTok}[1]{\textcolor[rgb]{0.68,0.00,0.00}{#1}}
\newcommand{\ExtensionTok}[1]{\textcolor[rgb]{0.00,0.23,0.31}{#1}}
\newcommand{\FloatTok}[1]{\textcolor[rgb]{0.68,0.00,0.00}{#1}}
\newcommand{\FunctionTok}[1]{\textcolor[rgb]{0.28,0.35,0.67}{#1}}
\newcommand{\ImportTok}[1]{\textcolor[rgb]{0.00,0.46,0.62}{#1}}
\newcommand{\InformationTok}[1]{\textcolor[rgb]{0.37,0.37,0.37}{#1}}
\newcommand{\KeywordTok}[1]{\textcolor[rgb]{0.00,0.23,0.31}{\textbf{#1}}}
\newcommand{\NormalTok}[1]{\textcolor[rgb]{0.00,0.23,0.31}{#1}}
\newcommand{\OperatorTok}[1]{\textcolor[rgb]{0.37,0.37,0.37}{#1}}
\newcommand{\OtherTok}[1]{\textcolor[rgb]{0.00,0.23,0.31}{#1}}
\newcommand{\PreprocessorTok}[1]{\textcolor[rgb]{0.68,0.00,0.00}{#1}}
\newcommand{\RegionMarkerTok}[1]{\textcolor[rgb]{0.00,0.23,0.31}{#1}}
\newcommand{\SpecialCharTok}[1]{\textcolor[rgb]{0.37,0.37,0.37}{#1}}
\newcommand{\SpecialStringTok}[1]{\textcolor[rgb]{0.13,0.47,0.30}{#1}}
\newcommand{\StringTok}[1]{\textcolor[rgb]{0.13,0.47,0.30}{#1}}
\newcommand{\VariableTok}[1]{\textcolor[rgb]{0.07,0.07,0.07}{#1}}
\newcommand{\VerbatimStringTok}[1]{\textcolor[rgb]{0.13,0.47,0.30}{#1}}
\newcommand{\WarningTok}[1]{\textcolor[rgb]{0.37,0.37,0.37}{\textit{#1}}}

\usepackage{longtable,booktabs,array}
\usepackage{calc} % for calculating minipage widths
% Correct order of tables after \paragraph or \subparagraph
\usepackage{etoolbox}
\makeatletter
\patchcmd\longtable{\par}{\if@noskipsec\mbox{}\fi\par}{}{}
\makeatother
% Allow footnotes in longtable head/foot
\IfFileExists{footnotehyper.sty}{\usepackage{footnotehyper}}{\usepackage{footnote}}
\makesavenoteenv{longtable}
\usepackage{graphicx}
\makeatletter
\newsavebox\pandoc@box
\newcommand*\pandocbounded[1]{% scales image to fit in text height/width
  \sbox\pandoc@box{#1}%
  \Gscale@div\@tempa{\textheight}{\dimexpr\ht\pandoc@box+\dp\pandoc@box\relax}%
  \Gscale@div\@tempb{\linewidth}{\wd\pandoc@box}%
  \ifdim\@tempb\p@<\@tempa\p@\let\@tempa\@tempb\fi% select the smaller of both
  \ifdim\@tempa\p@<\p@\scalebox{\@tempa}{\usebox\pandoc@box}%
  \else\usebox{\pandoc@box}%
  \fi%
}
% Set default figure placement to htbp
\def\fps@figure{htbp}
\makeatother





\setlength{\emergencystretch}{3em} % prevent overfull lines

\providecommand{\tightlist}{%
  \setlength{\itemsep}{0pt}\setlength{\parskip}{0pt}}



 


\KOMAoption{captions}{tableheading}
\makeatletter
\@ifpackageloaded{tcolorbox}{}{\usepackage[skins,breakable]{tcolorbox}}
\@ifpackageloaded{fontawesome5}{}{\usepackage{fontawesome5}}
\definecolor{quarto-callout-color}{HTML}{909090}
\definecolor{quarto-callout-note-color}{HTML}{0758E5}
\definecolor{quarto-callout-important-color}{HTML}{CC1914}
\definecolor{quarto-callout-warning-color}{HTML}{EB9113}
\definecolor{quarto-callout-tip-color}{HTML}{00A047}
\definecolor{quarto-callout-caution-color}{HTML}{FC5300}
\definecolor{quarto-callout-color-frame}{HTML}{acacac}
\definecolor{quarto-callout-note-color-frame}{HTML}{4582ec}
\definecolor{quarto-callout-important-color-frame}{HTML}{d9534f}
\definecolor{quarto-callout-warning-color-frame}{HTML}{f0ad4e}
\definecolor{quarto-callout-tip-color-frame}{HTML}{02b875}
\definecolor{quarto-callout-caution-color-frame}{HTML}{fd7e14}
\makeatother
\makeatletter
\@ifpackageloaded{caption}{}{\usepackage{caption}}
\AtBeginDocument{%
\ifdefined\contentsname
  \renewcommand*\contentsname{Table of contents}
\else
  \newcommand\contentsname{Table of contents}
\fi
\ifdefined\listfigurename
  \renewcommand*\listfigurename{List of Figures}
\else
  \newcommand\listfigurename{List of Figures}
\fi
\ifdefined\listtablename
  \renewcommand*\listtablename{List of Tables}
\else
  \newcommand\listtablename{List of Tables}
\fi
\ifdefined\figurename
  \renewcommand*\figurename{Figure}
\else
  \newcommand\figurename{Figure}
\fi
\ifdefined\tablename
  \renewcommand*\tablename{Table}
\else
  \newcommand\tablename{Table}
\fi
}
\@ifpackageloaded{float}{}{\usepackage{float}}
\floatstyle{ruled}
\@ifundefined{c@chapter}{\newfloat{codelisting}{h}{lop}}{\newfloat{codelisting}{h}{lop}[chapter]}
\floatname{codelisting}{Listing}
\newcommand*\listoflistings{\listof{codelisting}{List of Listings}}
\makeatother
\makeatletter
\makeatother
\makeatletter
\@ifpackageloaded{caption}{}{\usepackage{caption}}
\@ifpackageloaded{subcaption}{}{\usepackage{subcaption}}
\makeatother
\usepackage{bookmark}
\IfFileExists{xurl.sty}{\usepackage{xurl}}{} % add URL line breaks if available
\urlstyle{same}
\hypersetup{
  pdftitle={C++ 3rd Person Battery Collector Power Up Game},
  pdfauthor={DEPARTAMENTO DE ENGENHARIA INFORMÁTICA},
  colorlinks=true,
  linkcolor={blue},
  filecolor={Maroon},
  citecolor={Blue},
  urlcolor={Blue},
  pdfcreator={LaTeX via pandoc}}


\title{C++ 3rd Person Battery Collector Power Up Game}
\usepackage{etoolbox}
\makeatletter
\providecommand{\subtitle}[1]{% add subtitle to \maketitle
  \apptocmd{\@title}{\par {\large #1 \par}}{}{}
}
\makeatother
\subtitle{GAMES AND MULTIMEDIA - GAME ENGINES II}
\author{DEPARTAMENTO DE ENGENHARIA INFORMÁTICA}
\date{Invalid Date}
\begin{document}
\maketitle

\renewcommand*\contentsname{Table of contents}
{
\hypersetup{linkcolor=}
\setcounter{tocdepth}{3}
\tableofcontents
}

\section{Introduction}\label{introduction}

This assignment will introduce you to C++ programming and Unreal Engine
5 (UE5), including some of the Gameplay Framework classes. It will also
show you some of the different ways to work with classes, functions, and
variables that can help you with your project. This is just a jumping
off point to get you started, but it is a mini game with both winning
and losing.

As you will see, the character can collect batteries which are spawned
by a Spawn Volume above the level. Your power increases as you collect
the batteries, which changes your material and your running speed.
However, the GameMode is also draining your character over time. You can
lose power completely.

By the end of this assignment, you will have a number of classes
including:

\begin{itemize}
\tightlist
\item
  \textbf{Base pickup} - A battery pickup which fires lightning towards
  the character and gives some of its base power to the character
\item
  \textbf{Upgraded character} - Has a Collection Sphere to collect all
  of the batteries within a certain radius of the character
\item
  \textbf{GameMode} - Drains the character's power over time and defines
  what happens when you win and when you lose during gameplay
\item
  \textbf{Spawn Volume} - Spawns any kind of pickup whether it is a Base
  Pickup, a Battery Pickup, or some other kind of pickup that you
  develop on your own
\item
  \textbf{HUD} - Set up using UMG (Unreal Motion Graphics) and has the
  power bar and the text at the top of the screen
\end{itemize}

\section{Building the Base Level}\label{building-the-base-level}

\subsection{Setting Up the Project}\label{setting-up-the-project}

Open the Games tab in the Unreal Project Browser. We are going to use a
C++ template. Select \textbf{Third Person}. Leave the setup on Desktop.
You can change the quality to Maximum Quality or Scalable, depending on
your system setup.

\begin{tcolorbox}[enhanced jigsaw, arc=.35mm, toprule=.15mm, breakable, rightrule=.15mm, leftrule=.75mm, bottomtitle=1mm, left=2mm, title=\textcolor{quarto-callout-important-color}{\faExclamation}\hspace{0.5em}{Important}, bottomrule=.15mm, colframe=quarto-callout-important-color-frame, colback=white, coltitle=black, toptitle=1mm, titlerule=0mm, colbacktitle=quarto-callout-important-color!10!white, opacitybacktitle=0.6, opacityback=0]

The reason for not changing from Desktop to Mobile is that will change
some of the inputs for your project.

\end{tcolorbox}

You can either include Starter Content or not; we don't need it for this
assignment. However, if you want to take the project further, you can
add Starter Content now or add it later with the add Feature Packs in
the editor.

\textbf{Name the project ``BatteryCollector''}. If you name your project
something different, you will have to adjust the code from the
assignment. Click \textbf{Create}.

\subsection{Importing Content from Content
Examples}\label{importing-content-from-content-examples}

You need to use content from the Content Examples project. Download it
from the Epic Games Launcher.

There are 2 things you are going to use from the Content Examples
project: 1. A room for the character to run around 2. A battery mesh for
our battery pickup

\subsubsection{Migrating the Demo Room}\label{migrating-the-demo-room}

In the Content Drawer, search for ``BP\_DemoRoom''. This is the room we
are going to use.

\begin{enumerate}
\def\labelenumi{\arabic{enumi}.}
\tightlist
\item
  Select \textbf{BP\_DemoRoom} in the Content Drawer
\item
  Right-click, go to \textbf{Asset Actions} and select \textbf{Migrate}
\item
  Click \textbf{OK} when the asset list appears
\item
  Navigate to your project:
  \texttt{UnrealProjects\ \textgreater{}\ BatteryCollector\ \textgreater{}\ Content}
\item
  Click \textbf{Select Folder}
\end{enumerate}

\subsubsection{Migrating the Battery
Mesh}\label{migrating-the-battery-mesh}

Search for ``Battery''. Select the \textbf{SM\_Battery\_Medium}. 1.
Right-click, go to \textbf{Asset Actions}, and select \textbf{Migrate}
2. Click \textbf{OK} 3. The Content folder should still be selected,
click \textbf{OK}

\subsection{Setting Up the Level}\label{setting-up-the-level}

\begin{enumerate}
\def\labelenumi{\arabic{enumi}.}
\tightlist
\item
  Create a new level: \textbf{File → New Level → Basic}
\item
  Create a \textbf{Maps} folder within the Content folder
\item
  Save the level as ``\textbf{CollectionLevel}'' in the Maps folder
\end{enumerate}

\subsubsection{Setting Default Maps}\label{setting-default-maps}

\begin{enumerate}
\def\labelenumi{\arabic{enumi}.}
\tightlist
\item
  Click \textbf{Edit → Project Settings}
\item
  Select \textbf{Maps \& Modes}
\item
  Set both \textbf{Game Default Map} and \textbf{Editor Startup Map} to
  \textbf{CollectionLevel}
\end{enumerate}

\subsection{Level Design Setup}\label{level-design-setup}

\begin{enumerate}
\def\labelenumi{\arabic{enumi}.}
\tightlist
\item
  Delete the Floor from the World Outliner
\item
  Drag \textbf{BP\_DemoRoom} into the level
\item
  Configure the room properties:

  \begin{itemize}
  \tightlist
  \item
    Set \textbf{Width} to 32
  \item
    Uncheck \textbf{Pillar} and \textbf{Sub Pillar}
  \item
    Uncheck \textbf{Enclosed Left} (batteries will fall from the sky)
  \end{itemize}
\item
  Add a \textbf{Player Start} actor:

  \begin{itemize}
  \tightlist
  \item
    \textbf{Window → Place Actors}
  \item
    Search for \textbf{Player Start} and drag it to the center of the
    room
  \end{itemize}
\end{enumerate}

\section{Making Your First Pickup
Class}\label{making-your-first-pickup-class}

\subsection{Creating the Base Pickup
Class}\label{creating-the-base-pickup-class}

Use the C++ Class Wizard: \textbf{Tools → New C++ Class}

\begin{enumerate}
\def\labelenumi{\arabic{enumi}.}
\tightlist
\item
  Select \textbf{Actor} as the parent class
\item
  Name the class ``\textbf{Pickup}''
\item
  Click \textbf{Create Class}
\end{enumerate}

\subsection{Pickup Header File
(Pickup.h)}\label{pickup-header-file-pickup.h}

\begin{Shaded}
\begin{Highlighting}[]
\CommentTok{// Fill out your copyright notice in the Description page of Project Settings.}
\PreprocessorTok{\#pragma once}

\PreprocessorTok{\#include }\ImportTok{"CoreMinimal.h"}
\PreprocessorTok{\#include }\ImportTok{"GameFramework/Actor.h"}
\PreprocessorTok{\#include }\ImportTok{"Pickup.generated.h"}

\NormalTok{UCLASS}\OperatorTok{()}
\KeywordTok{class}\NormalTok{ BATTERYCOLLECTOR\_API APickup }\OperatorTok{:} \KeywordTok{public}\NormalTok{ AActor}
\OperatorTok{\{}
\NormalTok{    GENERATED\_BODY}\OperatorTok{()}
    
\KeywordTok{public}\OperatorTok{:}    
    \CommentTok{// Sets default values for this actor\textquotesingle{}s properties}
\NormalTok{    APickup}\OperatorTok{();}

\KeywordTok{protected}\OperatorTok{:}
    \CommentTok{// Called when the game starts or when spawned}
    \KeywordTok{virtual} \DataTypeTok{void}\NormalTok{ BeginPlay}\OperatorTok{()} \KeywordTok{override}\OperatorTok{;}
    
    \CommentTok{/** True when the pickup can be used, and false when the pickup is deactivated */}
    \DataTypeTok{bool}\NormalTok{ bIsActive}\OperatorTok{;}

\KeywordTok{public}\OperatorTok{:}    
    \CommentTok{// Called every frame}
    \KeywordTok{virtual} \DataTypeTok{void}\NormalTok{ Tick}\OperatorTok{(}\DataTypeTok{float}\NormalTok{ DeltaTime}\OperatorTok{)} \KeywordTok{override}\OperatorTok{;}
    
    \CommentTok{/** Return the mesh for the pickup */}
\NormalTok{    FORCEINLINE UStaticMeshComponent}\OperatorTok{*}\NormalTok{ GetMesh}\OperatorTok{()} \AttributeTok{const} \OperatorTok{\{} \ControlFlowTok{return}\NormalTok{ PickupMesh}\OperatorTok{;} \OperatorTok{\}}
    
    \CommentTok{/** Return whether or not the pickup is active */}
\NormalTok{    UFUNCTION}\OperatorTok{(}\NormalTok{BlueprintPure}\OperatorTok{,}\NormalTok{ Category }\OperatorTok{=} \StringTok{"Pickup"}\OperatorTok{)}
    \DataTypeTok{bool}\NormalTok{ IsActive}\OperatorTok{();}
    
    \CommentTok{/** Allows other classes to safely change whether or not pickup is active */}
\NormalTok{    UFUNCTION}\OperatorTok{(}\NormalTok{BlueprintCallable}\OperatorTok{,}\NormalTok{ Category }\OperatorTok{=} \StringTok{"Pickup"}\OperatorTok{)}
    \DataTypeTok{void}\NormalTok{ SetActive}\OperatorTok{(}\DataTypeTok{bool}\NormalTok{ NewPickupState}\OperatorTok{);}

\KeywordTok{private}\OperatorTok{:}
    \CommentTok{/** Static mesh to represent the pickup in the level */}
\NormalTok{    UPROPERTY}\OperatorTok{(}\NormalTok{VisibleAnywhere}\OperatorTok{,}\NormalTok{ BlueprintReadOnly}\OperatorTok{,}\NormalTok{ Category }\OperatorTok{=} \StringTok{"Pickup"}\OperatorTok{,}\NormalTok{ meta }\OperatorTok{=} \OperatorTok{(}\NormalTok{allowprivateaccess }\OperatorTok{=} \KeywordTok{true}\OperatorTok{))}
\NormalTok{    UStaticMeshComponent}\OperatorTok{*}\NormalTok{ PickupMesh}\OperatorTok{;}
\OperatorTok{\};}
\end{Highlighting}
\end{Shaded}

\subsection{Pickup Source File
(Pickup.cpp)}\label{pickup-source-file-pickup.cpp}

\begin{Shaded}
\begin{Highlighting}[]
\CommentTok{// Fill out your copyright notice in the Description page of Project Settings.}
\PreprocessorTok{\#include }\ImportTok{"Pickup.h"}

\CommentTok{// Sets default values}
\NormalTok{APickup}\OperatorTok{::}\NormalTok{APickup}\OperatorTok{()}
\OperatorTok{\{}
    \CommentTok{// Set this actor to call Tick() every frame. You can turn this off to improve performance if you don\textquotesingle{}t need it.}
\NormalTok{    PrimaryActorTick}\OperatorTok{.}\NormalTok{bCanEverTick }\OperatorTok{=} \KeywordTok{false}\OperatorTok{;}
    
    \CommentTok{// All pickups start active}
\NormalTok{    bIsActive }\OperatorTok{=} \KeywordTok{true}\OperatorTok{;}
    
    \CommentTok{// Create the static mesh component}
\NormalTok{    PickupMesh }\OperatorTok{=}\NormalTok{ CreateDefaultSubobject}\OperatorTok{\textless{}}\NormalTok{UStaticMeshComponent}\OperatorTok{\textgreater{}(}\NormalTok{TEXT}\OperatorTok{(}\StringTok{"PickupMesh"}\OperatorTok{));}
\NormalTok{    RootComponent }\OperatorTok{=}\NormalTok{ PickupMesh}\OperatorTok{;}
\OperatorTok{\}}

\CommentTok{// Called when the game starts or when spawned}
\DataTypeTok{void}\NormalTok{ APickup}\OperatorTok{::}\NormalTok{BeginPlay}\OperatorTok{()}
\OperatorTok{\{}
\NormalTok{    Super}\OperatorTok{::}\NormalTok{BeginPlay}\OperatorTok{();}
\OperatorTok{\}}

\CommentTok{// Called every frame}
\DataTypeTok{void}\NormalTok{ APickup}\OperatorTok{::}\NormalTok{Tick}\OperatorTok{(}\DataTypeTok{float}\NormalTok{ DeltaTime}\OperatorTok{)}
\OperatorTok{\{}
\NormalTok{    Super}\OperatorTok{::}\NormalTok{Tick}\OperatorTok{(}\NormalTok{DeltaTime}\OperatorTok{);}
\OperatorTok{\}}

\CommentTok{// Returns active state}
\DataTypeTok{bool}\NormalTok{ APickup}\OperatorTok{::}\NormalTok{IsActive}\OperatorTok{()}
\OperatorTok{\{}
    \ControlFlowTok{return}\NormalTok{ bIsActive}\OperatorTok{;}
\OperatorTok{\}}

\CommentTok{// Changes active state}
\DataTypeTok{void}\NormalTok{ APickup}\OperatorTok{::}\NormalTok{SetActive}\OperatorTok{(}\DataTypeTok{bool}\NormalTok{ NewPickupState}\OperatorTok{)}
\OperatorTok{\{}
\NormalTok{    bIsActive }\OperatorTok{=}\NormalTok{ NewPickupState}\OperatorTok{;}
\OperatorTok{\}}
\end{Highlighting}
\end{Shaded}

\subsection{Creating a Blueprint from the Pickup
Class}\label{creating-a-blueprint-from-the-pickup-class}

\begin{enumerate}
\def\labelenumi{\arabic{enumi}.}
\tightlist
\item
  Right-click on \textbf{Pickup} in C++ Classes
\item
  Select \textbf{Create Blueprint class based on Pickup}
\item
  Name it ``\textbf{Pickup\_BP}'' and save it in the Blueprints folder
\item
  Set the \textbf{Static Mesh} to \textbf{SM\_Cube}
\item
  Compile and Save
\end{enumerate}

\section{Extending the Pickup Class}\label{extending-the-pickup-class}

\subsection{Creating the Battery Pickup
Class}\label{creating-the-battery-pickup-class}

Create a new C++ class derived from Pickup:

\begin{enumerate}
\def\labelenumi{\arabic{enumi}.}
\tightlist
\item
  \textbf{Tools → New C++ Class}
\item
  Click \textbf{All Classes} and search for ``Pickup''
\item
  Select \textbf{Pickup} and click \textbf{Next}
\item
  Name the class ``\textbf{BatteryPickup}''
\end{enumerate}

\subsection{Battery Pickup Header File
(BatteryPickup.h)}\label{battery-pickup-header-file-batterypickup.h}

\begin{Shaded}
\begin{Highlighting}[]
\CommentTok{// Fill out your copyright notice in the Description page of Project Settings.}
\PreprocessorTok{\#pragma once}

\PreprocessorTok{\#include }\ImportTok{"CoreMinimal.h"}
\PreprocessorTok{\#include }\ImportTok{"Pickup.h"}
\PreprocessorTok{\#include }\ImportTok{"BatteryPickup.generated.h"}

\NormalTok{UCLASS}\OperatorTok{()}
\KeywordTok{class}\NormalTok{ BATTERYCOLLECTOR\_API ABatteryPickup }\OperatorTok{:} \KeywordTok{public}\NormalTok{ APickup}
\OperatorTok{\{}
\NormalTok{    GENERATED\_BODY}\OperatorTok{()}

\KeywordTok{public}\OperatorTok{:}
\NormalTok{    ABatteryPickup}\OperatorTok{();}
    
    \CommentTok{/** Override the WasCollected function {-} use Implementation because it\textquotesingle{}s a Blueprint Native Event */}
    \DataTypeTok{void}\NormalTok{ WasCollected\_Implementation}\OperatorTok{()} \KeywordTok{override}\OperatorTok{;}

\KeywordTok{protected}\OperatorTok{:}
    \CommentTok{/** Set the amount of power the battery gives to the character */}
\NormalTok{    UPROPERTY}\OperatorTok{(}\NormalTok{EditAnywhere}\OperatorTok{,}\NormalTok{ BlueprintReadWrite}\OperatorTok{,}\NormalTok{ Category }\OperatorTok{=} \StringTok{"Power"}\OperatorTok{,}\NormalTok{ Meta }\OperatorTok{=} \OperatorTok{(}\NormalTok{BlueprintProtected }\OperatorTok{=} \StringTok{"true"}\OperatorTok{))}
    \DataTypeTok{float}\NormalTok{ BatteryPower}\OperatorTok{;}
    
\KeywordTok{public}\OperatorTok{:}
    \CommentTok{/** Public way to access the battery\textquotesingle{}s power level */}
    \DataTypeTok{float}\NormalTok{ GetPower}\OperatorTok{();}
\OperatorTok{\};}
\end{Highlighting}
\end{Shaded}

\subsection{Battery Pickup Source File
(BatteryPickup.cpp)}\label{battery-pickup-source-file-batterypickup.cpp}

\begin{Shaded}
\begin{Highlighting}[]
\CommentTok{// Fill out your copyright notice in the Description page of Project Settings.}
\PreprocessorTok{\#include }\ImportTok{"BatteryPickup.h"}

\NormalTok{ABatteryPickup}\OperatorTok{::}\NormalTok{ABatteryPickup}\OperatorTok{()}
\OperatorTok{\{}
\NormalTok{    GetMesh}\OperatorTok{(){-}\textgreater{}}\NormalTok{SetSimulatePhysics}\OperatorTok{(}\KeywordTok{true}\OperatorTok{);}
    
    \CommentTok{// The base power level of the battery}
\NormalTok{    BatteryPower }\OperatorTok{=} \FloatTok{150.}\BuiltInTok{f}\OperatorTok{;}
\OperatorTok{\}}

\DataTypeTok{void}\NormalTok{ ABatteryPickup}\OperatorTok{::}\NormalTok{WasCollected\_Implementation}\OperatorTok{()}
\OperatorTok{\{}
    \CommentTok{// Use the base pickup behavior}
\NormalTok{    Super}\OperatorTok{::}\NormalTok{WasCollected\_Implementation}\OperatorTok{();}
    
    \CommentTok{// Destroy the battery}
\NormalTok{    Destroy}\OperatorTok{();}
\OperatorTok{\}}

\CommentTok{// Report the power level of the battery}
\DataTypeTok{float}\NormalTok{ ABatteryPickup}\OperatorTok{::}\NormalTok{GetPower}\OperatorTok{()}
\OperatorTok{\{}
    \ControlFlowTok{return}\NormalTok{ BatteryPower}\OperatorTok{;}
\OperatorTok{\}}
\end{Highlighting}
\end{Shaded}

\subsection{Setting Up Battery
Collision}\label{setting-up-battery-collision}

The battery needs collision to stop falling through the floor:

\begin{enumerate}
\def\labelenumi{\arabic{enumi}.}
\tightlist
\item
  Open \textbf{SM\_Battery\_Medium} in the Static Mesh Editor
\item
  Click \textbf{Collision → Add 18DOP Simplified Collision}
\item
  Save the mesh
\end{enumerate}

\subsection{Creating Battery
Blueprint}\label{creating-battery-blueprint}

\begin{enumerate}
\def\labelenumi{\arabic{enumi}.}
\tightlist
\item
  Right-click \textbf{BatteryPickup} and select \textbf{Create Blueprint
  class based on BatteryPickup}
\item
  Name it ``\textbf{Battery\_BP}''
\item
  Set the \textbf{Static Mesh} to \textbf{SM\_Battery\_Medium}
\item
  Compile and Save
\end{enumerate}

\section{Creating the Spawning
Volume}\label{creating-the-spawning-volume}

\subsection{Spawn Volume Class}\label{spawn-volume-class}

Create a new Actor class called ``\textbf{SpawnVolume}''.

\subsection{Spawn Volume Header File
(SpawnVolume.h)}\label{spawn-volume-header-file-spawnvolume.h}

\begin{Shaded}
\begin{Highlighting}[]
\PreprocessorTok{\#include }\ImportTok{"CoreMinimal.h"}
\PreprocessorTok{\#include }\ImportTok{"GameFramework/Actor.h"}
\PreprocessorTok{\#include }\ImportTok{\textless{}Components/BoxComponent.h\textgreater{}}
\PreprocessorTok{\#include }\ImportTok{\textless{}Kismet/KismetMathLibrary.h\textgreater{}}
\PreprocessorTok{\#include }\ImportTok{"Pickup.h"}
\PreprocessorTok{\#include }\ImportTok{"SpawnVolume.generated.h"}

\NormalTok{UCLASS}\OperatorTok{()}
\KeywordTok{class}\NormalTok{ BATTERYCOLLECTOR\_API ASpawnVolume }\OperatorTok{:} \KeywordTok{public}\NormalTok{ AActor}
\OperatorTok{\{}
\NormalTok{    GENERATED\_BODY}\OperatorTok{()}
    
\KeywordTok{public}\OperatorTok{:}    
    \CommentTok{// Sets default values for this actor\textquotesingle{}s properties}
\NormalTok{    ASpawnVolume}\OperatorTok{();}

\KeywordTok{protected}\OperatorTok{:}
    \CommentTok{// Called when the game starts or when spawned}
    \KeywordTok{virtual} \DataTypeTok{void}\NormalTok{ BeginPlay}\OperatorTok{()} \KeywordTok{override}\OperatorTok{;}
    
    \CommentTok{/** The pickup to spawn */}
\NormalTok{    UPROPERTY}\OperatorTok{(}\NormalTok{EditAnywhere}\OperatorTok{,}\NormalTok{ Category }\OperatorTok{=} \StringTok{"Spawning"}\OperatorTok{)}
\NormalTok{    TSubclassOf}\OperatorTok{\textless{}}\KeywordTok{class}\NormalTok{ APickup}\OperatorTok{\textgreater{}}\NormalTok{ WhatToSpawn}\OperatorTok{;}
    
\NormalTok{    FTimerHandle SpawnTimer}\OperatorTok{;}
    
    \CommentTok{/** Minimum spawn delay */}
\NormalTok{    UPROPERTY}\OperatorTok{(}\NormalTok{EditAnywhere}\OperatorTok{,}\NormalTok{ BlueprintReadWrite}\OperatorTok{,}\NormalTok{ Category }\OperatorTok{=} \StringTok{"Spawning"}\OperatorTok{)}
    \DataTypeTok{float}\NormalTok{ SpawnDelayRangeLow}\OperatorTok{;}
    
    \CommentTok{/** Maximum spawn delay */}
\NormalTok{    UPROPERTY}\OperatorTok{(}\NormalTok{EditAnywhere}\OperatorTok{,}\NormalTok{ BlueprintReadWrite}\OperatorTok{,}\NormalTok{ Category }\OperatorTok{=} \StringTok{"Spawning"}\OperatorTok{)}
    \DataTypeTok{float}\NormalTok{ SpawnDelayRangeHigh}\OperatorTok{;}

\KeywordTok{public}\OperatorTok{:}    
    \CommentTok{// Called every frame}
    \KeywordTok{virtual} \DataTypeTok{void}\NormalTok{ Tick}\OperatorTok{(}\DataTypeTok{float}\NormalTok{ DeltaTime}\OperatorTok{)} \KeywordTok{override}\OperatorTok{;}
    
    \CommentTok{/** Returns the WhereToSpawn subobject */}
\NormalTok{    FORCEINLINE }\KeywordTok{class}\NormalTok{ UBoxComponent}\OperatorTok{*}\NormalTok{ GetWhereToSpawn}\OperatorTok{()} \AttributeTok{const} \OperatorTok{\{} \ControlFlowTok{return}\NormalTok{ WhereToSpawn}\OperatorTok{;} \OperatorTok{\}}
    
    \CommentTok{/** Find a random point within the BoxComponent */}
\NormalTok{    UFUNCTION}\OperatorTok{(}\NormalTok{BlueprintPure}\OperatorTok{,}\NormalTok{ Category }\OperatorTok{=} \StringTok{"Spawning"}\OperatorTok{)}
\NormalTok{    FVector GetRandomPointInVolume}\OperatorTok{();}
    
    \CommentTok{/** This function toggles whether or not the spawn volume spawns pickups */}
\NormalTok{    UFUNCTION}\OperatorTok{(}\NormalTok{BlueprintCallable}\OperatorTok{,}\NormalTok{ Category }\OperatorTok{=} \StringTok{"Spawning"}\OperatorTok{)}
    \DataTypeTok{void}\NormalTok{ SetSpawningActive}\OperatorTok{(}\DataTypeTok{bool}\NormalTok{ bShouldSpawn}\OperatorTok{);}

\KeywordTok{private}\OperatorTok{:}
    \CommentTok{/** Box Component to specify where the pickups should be spawned */}
\NormalTok{    UPROPERTY}\OperatorTok{(}\NormalTok{VisibleAnywhere}\OperatorTok{,}\NormalTok{ BlueprintReadOnly}\OperatorTok{,}\NormalTok{ Category }\OperatorTok{=} \StringTok{"Spawning"}\OperatorTok{,}\NormalTok{ meta }\OperatorTok{=} \OperatorTok{(}\NormalTok{AllowPrivateAccess }\OperatorTok{=} \StringTok{"true"}\OperatorTok{))}
\NormalTok{    UBoxComponent}\OperatorTok{*}\NormalTok{ WhereToSpawn}\OperatorTok{;}
    
    \CommentTok{/** Handle spawning a new pickup */}
    \DataTypeTok{void}\NormalTok{ SpawnPickup}\OperatorTok{();}
    
    \CommentTok{/** The current spawn delay */}
    \DataTypeTok{float}\NormalTok{ SpawnDelay}\OperatorTok{;}
\OperatorTok{\};}
\end{Highlighting}
\end{Shaded}

\subsection{Spawn Volume Source File
(SpawnVolume.cpp)}\label{spawn-volume-source-file-spawnvolume.cpp}

\begin{Shaded}
\begin{Highlighting}[]
\PreprocessorTok{\#include }\ImportTok{"SpawnVolume.h"}
\PreprocessorTok{\#include }\ImportTok{\textless{}Kismet/GameplayStatics.h\textgreater{}}

\CommentTok{// Sets default values}
\NormalTok{ASpawnVolume}\OperatorTok{::}\NormalTok{ASpawnVolume}\OperatorTok{()}
\OperatorTok{\{}
    \CommentTok{// Set this actor to call Tick() every frame. You can turn this off to improve performance if you don\textquotesingle{}t need it.}
\NormalTok{    PrimaryActorTick}\OperatorTok{.}\NormalTok{bCanEverTick }\OperatorTok{=} \KeywordTok{false}\OperatorTok{;}
    
    \CommentTok{// Create the Box Component to represent the spawn volume}
\NormalTok{    WhereToSpawn }\OperatorTok{=}\NormalTok{ CreateDefaultSubobject}\OperatorTok{\textless{}}\NormalTok{UBoxComponent}\OperatorTok{\textgreater{}(}\NormalTok{TEXT}\OperatorTok{(}\StringTok{"WhereToSpawn"}\OperatorTok{));}
\NormalTok{    RootComponent }\OperatorTok{=}\NormalTok{ WhereToSpawn}\OperatorTok{;}
    
    \CommentTok{// Set the spawn delay range}
\NormalTok{    SpawnDelayRangeLow }\OperatorTok{=} \FloatTok{1.0}\BuiltInTok{f}\OperatorTok{;}
\NormalTok{    SpawnDelayRangeHigh }\OperatorTok{=} \FloatTok{4.5}\BuiltInTok{f}\OperatorTok{;}
\OperatorTok{\}}

\CommentTok{// Called when the game starts or when spawned}
\DataTypeTok{void}\NormalTok{ ASpawnVolume}\OperatorTok{::}\NormalTok{BeginPlay}\OperatorTok{()}
\OperatorTok{\{}
\NormalTok{    Super}\OperatorTok{::}\NormalTok{BeginPlay}\OperatorTok{();}
\OperatorTok{\}}

\CommentTok{// Called every frame}
\DataTypeTok{void}\NormalTok{ ASpawnVolume}\OperatorTok{::}\NormalTok{Tick}\OperatorTok{(}\DataTypeTok{float}\NormalTok{ DeltaTime}\OperatorTok{)}
\OperatorTok{\{}
\NormalTok{    Super}\OperatorTok{::}\NormalTok{Tick}\OperatorTok{(}\NormalTok{DeltaTime}\OperatorTok{);}
\OperatorTok{\}}

\NormalTok{FVector ASpawnVolume}\OperatorTok{::}\NormalTok{GetRandomPointInVolume}\OperatorTok{()}
\OperatorTok{\{}
\NormalTok{    FVector SpawnOrigin }\OperatorTok{=}\NormalTok{ WhereToSpawn}\OperatorTok{{-}\textgreater{}}\NormalTok{Bounds}\OperatorTok{.}\NormalTok{Origin}\OperatorTok{;}
\NormalTok{    FVector SpawnExtent }\OperatorTok{=}\NormalTok{ WhereToSpawn}\OperatorTok{{-}\textgreater{}}\NormalTok{Bounds}\OperatorTok{.}\NormalTok{BoxExtent}\OperatorTok{;}
    
    \ControlFlowTok{return}\NormalTok{ UKismetMathLibrary}\OperatorTok{::}\NormalTok{RandomPointInBoundingBox}\OperatorTok{(}\NormalTok{SpawnOrigin}\OperatorTok{,}\NormalTok{ SpawnExtent}\OperatorTok{);}
\OperatorTok{\}}

\DataTypeTok{void}\NormalTok{ ASpawnVolume}\OperatorTok{::}\NormalTok{SetSpawningActive}\OperatorTok{(}\DataTypeTok{bool}\NormalTok{ bShouldSpawn}\OperatorTok{)}
\OperatorTok{\{}
    \ControlFlowTok{if} \OperatorTok{(}\NormalTok{bShouldSpawn}\OperatorTok{)}
    \OperatorTok{\{}
        \CommentTok{// Set the timer on Spawn Pickup}
\NormalTok{        SpawnDelay }\OperatorTok{=}\NormalTok{ FMath}\OperatorTok{::}\NormalTok{FRandRange}\OperatorTok{(}\NormalTok{SpawnDelayRangeLow}\OperatorTok{,}\NormalTok{ SpawnDelayRangeHigh}\OperatorTok{);}
\NormalTok{        GetWorldTimerManager}\OperatorTok{().}\NormalTok{SetTimer}\OperatorTok{(}\NormalTok{SpawnTimer}\OperatorTok{,} \KeywordTok{this}\OperatorTok{,} \OperatorTok{\&}\NormalTok{ASpawnVolume}\OperatorTok{::}\NormalTok{SpawnPickup}\OperatorTok{,}\NormalTok{ SpawnDelay}\OperatorTok{,} \KeywordTok{false}\OperatorTok{);}
    \OperatorTok{\}}
    \ControlFlowTok{else}
    \OperatorTok{\{}
        \CommentTok{// Clear the timer on Spawn Pickup}
\NormalTok{        GetWorldTimerManager}\OperatorTok{().}\NormalTok{ClearTimer}\OperatorTok{(}\NormalTok{SpawnTimer}\OperatorTok{);}
    \OperatorTok{\}}
\OperatorTok{\}}

\DataTypeTok{void}\NormalTok{ ABatteryCollectorGameMode}\OperatorTok{::}\NormalTok{Tick}\OperatorTok{(}\DataTypeTok{float}\NormalTok{ DeltaTime}\OperatorTok{)}
\OperatorTok{\{}
\NormalTok{    Super}\OperatorTok{::}\NormalTok{Tick}\OperatorTok{(}\NormalTok{DeltaTime}\OperatorTok{);}
    
    \CommentTok{// Check that we are using the battery collector character }
\NormalTok{    ABatteryCollectorCharacter}\OperatorTok{*}\NormalTok{ MyCharacter }\OperatorTok{=}\NormalTok{ Cast}\OperatorTok{\textless{}}\NormalTok{ABatteryCollectorCharacter}\OperatorTok{\textgreater{}(}\NormalTok{UGameplayStatics}\OperatorTok{::}\NormalTok{GetPlayerPawn}\OperatorTok{(}\KeywordTok{this}\OperatorTok{,} \DecValTok{0}\OperatorTok{));}
    \ControlFlowTok{if} \OperatorTok{(}\NormalTok{MyCharacter}\OperatorTok{)}
    \OperatorTok{\{}
        \CommentTok{// If our power is greater than needed to win, set the game\textquotesingle{}s state to Won}
        \ControlFlowTok{if} \OperatorTok{(}\NormalTok{MyCharacter}\OperatorTok{{-}\textgreater{}}\NormalTok{GetCurrentPower}\OperatorTok{()} \OperatorTok{\textgreater{}}\NormalTok{ PowerToWin}\OperatorTok{)}
        \OperatorTok{\{}
\NormalTok{            SetCurrentState}\OperatorTok{(}\NormalTok{EBatteryPlayState}\OperatorTok{::}\NormalTok{EWon}\OperatorTok{);}
        \OperatorTok{\}}
        \CommentTok{// If the character\textquotesingle{}s power is positive}
        \ControlFlowTok{else} \ControlFlowTok{if} \OperatorTok{(}\NormalTok{MyCharacter}\OperatorTok{{-}\textgreater{}}\NormalTok{GetCurrentPower}\OperatorTok{()} \OperatorTok{\textgreater{}} \DecValTok{0}\OperatorTok{)}
        \OperatorTok{\{}
            \CommentTok{// Decrease the character\textquotesingle{}s power using the decay rate}
\NormalTok{            MyCharacter}\OperatorTok{{-}\textgreater{}}\NormalTok{UpdatePower}\OperatorTok{({-}}\NormalTok{DeltaTime }\OperatorTok{*}\NormalTok{ DecayRate }\OperatorTok{*} \OperatorTok{(}\NormalTok{MyCharacter}\OperatorTok{{-}\textgreater{}}\NormalTok{GetInitialPower}\OperatorTok{()));}
        \OperatorTok{\}}
        \ControlFlowTok{else}
        \OperatorTok{\{}
\NormalTok{            SetCurrentState}\OperatorTok{(}\NormalTok{EBatteryPlayState}\OperatorTok{::}\NormalTok{EGameOver}\OperatorTok{);}
        \OperatorTok{\}}
    \OperatorTok{\}}
\OperatorTok{\}}
\end{Highlighting}
\end{Shaded}

\section{Final Setup and Polish}\label{final-setup-and-polish}

\subsection{Creating GameMode
Blueprint}\label{creating-gamemode-blueprint}

\begin{enumerate}
\def\labelenumi{\arabic{enumi}.}
\tightlist
\item
  Create a \textbf{GameMode\_BP} Blueprint based on
  \textbf{BatteryCollectorGameMode}
\item
  Set \textbf{HUDWidget Class} to \textbf{BatteryHUD}
\item
  Set this as the \textbf{Default GameMode} in \textbf{Project Settings
  → Maps \& Modes}
\end{enumerate}

\subsection{Character Collision Setup}\label{character-collision-setup}

In \textbf{BP\_ThirdPersonCharacter}:

\begin{enumerate}
\def\labelenumi{\arabic{enumi}.}
\tightlist
\item
  Select the \textbf{Mesh} component
\item
  Set \textbf{Collision Presets} to \textbf{Ragdoll} (so the character
  collides when ragdolled)
\end{enumerate}

\subsection{Level Polish}\label{level-polish}

\subsubsection{Spawn Volume Setup}\label{spawn-volume-setup}

\begin{enumerate}
\def\labelenumi{\arabic{enumi}.}
\tightlist
\item
  Position the spawn volume above the room
\item
  Scale it to cover most of the playing area
\item
  Set appropriate spawn delay ranges
\item
  Ensure \textbf{What to Spawn} is set to \textbf{Battery\_BP}
\end{enumerate}

\subsubsection{Lighting and Visual
Polish}\label{lighting-and-visual-polish}

\begin{enumerate}
\def\labelenumi{\arabic{enumi}.}
\tightlist
\item
  Build lighting for the final room setup
\item
  Adjust \textbf{Decay Rate} in \textbf{GameMode\_BP} for balanced
  gameplay
\item
  Test win/lose conditions
\end{enumerate}

\subsection{HUD Text Styling}\label{hud-text-styling}

In \textbf{BatteryHUD}:

\begin{enumerate}
\def\labelenumi{\arabic{enumi}.}
\tightlist
\item
  Increase \textbf{Font Size} to 48 for better visibility
\item
  Set up proper anchoring for responsive design
\item
  Ensure text is centered and visible
\end{enumerate}

\section{Testing the Complete Game}\label{testing-the-complete-game}

The completed game should have:

\begin{itemize}
\tightlist
\item
  \textbf{Character movement} and collection mechanics
\item
  \textbf{Power system} with visual feedback (color changes, speed
  changes)
\item
  \textbf{Battery spawning} from spawn volumes
\item
  \textbf{Electric arc effects} when collecting batteries
\item
  \textbf{HUD} showing power level and game state
\item
  \textbf{Win condition} when power exceeds PowerToWin threshold
\item
  \textbf{Lose condition} when power reaches zero (with ragdoll effect)
\item
  \textbf{Game state management} that stops spawning and disables input
  appropriately
\end{itemize}

\section{Exercises}\label{exercises}

Using what you have learned throughout this assignment, try to do the
following:

\subsection{Basic Extensions}\label{basic-extensions}

\begin{enumerate}
\def\labelenumi{\arabic{enumi}.}
\tightlist
\item
  \textbf{Character Glow Effect}

  \begin{itemize}
  \tightlist
  \item
    Make your Character glow as you collect batteries
  \item
    Hint: Use emissive materials or particle effects
  \end{itemize}
\item
  \textbf{Battery Counter}

  \begin{itemize}
  \tightlist
  \item
    Show the number of batteries you have collected on a counter in the
    corner
  \item
    Add a variable to track collected battery count
  \end{itemize}
\item
  \textbf{Night Scene}

  \begin{itemize}
  \tightlist
  \item
    Make it a nighttime scene by playing with the sky sphere and the
    directional light
  \item
    Experiment with atmospheric lighting
  \end{itemize}
\end{enumerate}

\subsection{Intermediate Extensions}\label{intermediate-extensions}

\begin{enumerate}
\def\labelenumi{\arabic{enumi}.}
\setcounter{enumi}{3}
\tightlist
\item
  \textbf{Red Battery Variant}

  \begin{itemize}
  \tightlist
  \item
    Create a copy of the original SM\_Battery\_Medium mesh
  \item
    Change its material to become red instead of orange
  \item
    Create a new type of Battery with 300 power
  \end{itemize}
\item
  \textbf{Special Pickup System}

  \begin{itemize}
  \tightlist
  \item
    Create a new type of Spawn Volume with probability-based spawning
  \item
    Add a probability parameter (default value = 0.1) for special
    battery pickups
  \item
    Set the red battery as the special type of pickup
  \end{itemize}
\end{enumerate}

\subsection{Advanced Extensions}\label{advanced-extensions}

\begin{enumerate}
\def\labelenumi{\arabic{enumi}.}
\setcounter{enumi}{5}
\tightlist
\item
  \textbf{Enhanced Spawn Volume}

  \begin{itemize}
  \tightlist
  \item
    Create a spawn volume that can spawn multiple pickup types
  \item
    Implement weighted random selection for different pickup types
  \item
    Add configuration for spawn probabilities per pickup type
  \end{itemize}
\end{enumerate}

\section{Troubleshooting}\label{troubleshooting}

\subsection{Common Issues}\label{common-issues}

\begin{tcolorbox}[enhanced jigsaw, arc=.35mm, toprule=.15mm, breakable, rightrule=.15mm, leftrule=.75mm, bottomtitle=1mm, left=2mm, title=\textcolor{quarto-callout-warning-color}{\faExclamationTriangle}\hspace{0.5em}{Warning}, bottomrule=.15mm, colframe=quarto-callout-warning-color-frame, colback=white, coltitle=black, toptitle=1mm, titlerule=0mm, colbacktitle=quarto-callout-warning-color!10!white, opacitybacktitle=0.6, opacityback=0]

\textbf{Batteries Falling Through Floor} - Ensure collision is set up on
SM\_Battery\_Medium - Use 18DOP Simplified Collision for better rolling
behavior

\end{tcolorbox}

\begin{tcolorbox}[enhanced jigsaw, arc=.35mm, toprule=.15mm, breakable, rightrule=.15mm, leftrule=.75mm, bottomtitle=1mm, left=2mm, title=\textcolor{quarto-callout-warning-color}{\faExclamationTriangle}\hspace{0.5em}{Warning}, bottomrule=.15mm, colframe=quarto-callout-warning-color-frame, colback=white, coltitle=black, toptitle=1mm, titlerule=0mm, colbacktitle=quarto-callout-warning-color!10!white, opacitybacktitle=0.6, opacityback=0]

\textbf{Character Not Ragdolling} - Check that Mesh collision is set to
``Ragdoll'' preset - Ensure SetSimulatePhysics(true) is being called

\end{tcolorbox}

\begin{tcolorbox}[enhanced jigsaw, arc=.35mm, toprule=.15mm, breakable, rightrule=.15mm, leftrule=.75mm, bottomtitle=1mm, left=2mm, title=\textcolor{quarto-callout-warning-color}{\faExclamationTriangle}\hspace{0.5em}{Warning}, bottomrule=.15mm, colframe=quarto-callout-warning-color-frame, colback=white, coltitle=black, toptitle=1mm, titlerule=0mm, colbacktitle=quarto-callout-warning-color!10!white, opacitybacktitle=0.6, opacityback=0]

\textbf{Spawn Volumes Not Working} - Verify GameMode is finding and
registering spawn volumes - Check that SetSpawningActive(true) is being
called in EPlaying state - Ensure WhatToSpawn is set to a valid pickup
class

\end{tcolorbox}

\subsection{Performance
Considerations}\label{performance-considerations}

\begin{itemize}
\tightlist
\item
  \textbf{Spawn Rate Balance}: Too many spawned pickups can affect
  performance
\item
  \textbf{Particle Effects}: Electric arcs create temporary particle
  systems
\item
  \textbf{Collection Sphere}: Large radii may impact collision detection
  performance
\end{itemize}

\section{Conclusion}\label{conclusion}

This assignment demonstrates fundamental game development concepts in
Unreal Engine 5 with C++:

\begin{itemize}
\tightlist
\item
  \textbf{Object-Oriented Design}: Inheritance hierarchy with base and
  derived pickup classes
\item
  \textbf{Component-Based Architecture}: Using UE5's component system
  for modular functionality
\item
  \textbf{Game State Management}: Implementing different game states
  with appropriate transitions
\item
  \textbf{UI Integration}: Creating responsive HUDs with UMG
\item
  \textbf{Physics Integration}: Combining physics simulation with
  gameplay mechanics
\item
  \textbf{Visual Effects}: Particle systems and dynamic material changes
\item
  \textbf{Input Handling}: Enhanced Input System integration
\end{itemize}

The modular design allows for easy extension - new pickup types,
different game modes, and additional mechanics can be added by following
the established patterns.

\section{Source Reference}\label{source-reference}

\textbf{Source}:
\url{https://www.youtube.com/playlist?list=PLZlv_N0_O1gYup-gvJtMsgJqnEB_dGiM4}

\begin{center}\rule{0.5\linewidth}{0.5pt}\end{center}

\emph{This document was converted from the original PDF assignment for
the Games and Multimedia - Game Engines II course, Academic Year
2022/2023, 2nd Semester.}

void ASpawnVolume::SpawnPickup() \{ // If we have something to spawn: if
(WhatToSpawn != nullptr) \{ // Check for a valid World: UWorld* const
World = GetWorld(); if (World) \{ // Set the spawn parameters
FActorSpawnParameters SpawnParams; SpawnParams.Owner = this;
SpawnParams.Instigator = GetInstigator();

\begin{verbatim}
        // Get a random location to spawn at
        FVector SpawnLocation = GetRandomPointInVolume();
        
        // Get a random rotation for the spawned item
        FRotator SpawnRotation;
        SpawnRotation.Yaw = FMath::FRand() * 360.0f;
        SpawnRotation.Pitch = FMath::FRand() * 360.0f;
        SpawnRotation.Roll = FMath::FRand() * 360.0f;
        
        // Spawn the pickup
        APickup* const SpawnedPickup = World->SpawnActor<APickup>(WhatToSpawn, SpawnLocation, SpawnRotation, SpawnParams);
        
        SpawnDelay = FMath::FRandRange(SpawnDelayRangeLow, SpawnDelayRangeHigh);
        GetWorldTimerManager().SetTimer(SpawnTimer, this, &ASpawnVolume::SpawnPickup, SpawnDelay, false);
    }
}
\end{verbatim}

\}

\begin{verbatim}

### Setting Up Spawn Volumes in Level

1. Drag **SpawnVolume** directly into your level
2. Scale it appropriately using the **R** key
3. In the Details panel, set **What to Spawn** to **Battery_BP**
4. Adjust **Spawn Delay Range Low** and **High** values as needed

## Extending the Character Class

### Adding Collection Sphere

In **BatteryCollectorCharacter.h**, add the collection sphere:

```cpp
/** Collection sphere */
UPROPERTY(VisibleAnywhere, BlueprintReadOnly, Category = Pickups, meta = (AllowPrivateAccess = "true"))
class USphereComponent* CollectionSphere;

/** Return CollectionSphere subobject **/
FORCEINLINE class USphereComponent* GetCollectionSphere() const { return CollectionSphere; }
\end{verbatim}

In \textbf{BatteryCollectorCharacter.cpp} constructor:

\begin{Shaded}
\begin{Highlighting}[]
\CommentTok{// Create the collection sphere}
\NormalTok{CollectionSphere }\OperatorTok{=}\NormalTok{ CreateDefaultSubobject}\OperatorTok{\textless{}}\NormalTok{USphereComponent}\OperatorTok{\textgreater{}(}\NormalTok{TEXT}\OperatorTok{(}\StringTok{"CollectionSphere"}\OperatorTok{));}
\NormalTok{CollectionSphere}\OperatorTok{{-}\textgreater{}}\NormalTok{SetupAttachment}\OperatorTok{(}\NormalTok{RootComponent}\OperatorTok{);}
\NormalTok{CollectionSphere}\OperatorTok{{-}\textgreater{}}\NormalTok{SetSphereRadius}\OperatorTok{(}\FloatTok{200.}\BuiltInTok{f}\OperatorTok{);}
\end{Highlighting}
\end{Shaded}

\subsection{Adding Collection
Functions}\label{adding-collection-functions}

Add to the header file:

\begin{Shaded}
\begin{Highlighting}[]
\CommentTok{/** Called when we press a key to collect any pickups inside the CollectionSphere */}
\NormalTok{UFUNCTION}\OperatorTok{(}\NormalTok{BlueprintCallable}\OperatorTok{,}\NormalTok{ Category }\OperatorTok{=} \StringTok{"Pickups"}\OperatorTok{)}
\DataTypeTok{void}\NormalTok{ CollectPickups}\OperatorTok{();}
\end{Highlighting}
\end{Shaded}

\subsection{Adding Pickup Collection
Logic}\label{adding-pickup-collection-logic}

In \textbf{BatteryCollectorCharacter.cpp}:

\begin{Shaded}
\begin{Highlighting}[]
\PreprocessorTok{\#include }\ImportTok{"Pickup.h"}

\DataTypeTok{void}\NormalTok{ ABatteryCollectorCharacter}\OperatorTok{::}\NormalTok{CollectPickups}\OperatorTok{()}
\OperatorTok{\{}
    \CommentTok{// Get all overlapping Actors and store them in an array}
\NormalTok{    TArray}\OperatorTok{\textless{}}\NormalTok{AActor}\OperatorTok{*\textgreater{}}\NormalTok{ CollectedActors}\OperatorTok{;}
\NormalTok{    CollectionSphere}\OperatorTok{{-}\textgreater{}}\NormalTok{GetOverlappingActors}\OperatorTok{(}\NormalTok{CollectedActors}\OperatorTok{);}
    
    \CommentTok{// For each actor we collected}
    \ControlFlowTok{for} \OperatorTok{(}\NormalTok{int32 iCollected }\OperatorTok{=} \DecValTok{0}\OperatorTok{;}\NormalTok{ iCollected }\OperatorTok{\textless{}}\NormalTok{ CollectedActors}\OperatorTok{.}\NormalTok{Num}\OperatorTok{();} \OperatorTok{++}\NormalTok{iCollected}\OperatorTok{)}
    \OperatorTok{\{}
        \CommentTok{// Cast the actor to APickup}
\NormalTok{        APickup}\OperatorTok{*} \AttributeTok{const}\NormalTok{ TestPickup }\OperatorTok{=}\NormalTok{ Cast}\OperatorTok{\textless{}}\NormalTok{APickup}\OperatorTok{\textgreater{}(}\NormalTok{CollectedActors}\OperatorTok{[}\NormalTok{iCollected}\OperatorTok{]);}
        
        \CommentTok{// If the cast is successful and the pickup is valid and active}
        \ControlFlowTok{if} \OperatorTok{(}\NormalTok{TestPickup }\OperatorTok{\&\&}\NormalTok{ IsValid}\OperatorTok{(}\NormalTok{TestPickup}\OperatorTok{)} \OperatorTok{\&\&}\NormalTok{ TestPickup}\OperatorTok{{-}\textgreater{}}\NormalTok{IsActive}\OperatorTok{())}
        \OperatorTok{\{}
            \CommentTok{// Call the pickup\textquotesingle{}s WasCollected function}
\NormalTok{            TestPickup}\OperatorTok{{-}\textgreater{}}\NormalTok{WasCollected}\OperatorTok{();}
            
            \CommentTok{// Deactivate the pickup}
\NormalTok{            TestPickup}\OperatorTok{{-}\textgreater{}}\NormalTok{SetActive}\OperatorTok{(}\KeywordTok{false}\OperatorTok{);}
        \OperatorTok{\}}
    \OperatorTok{\}}
\OperatorTok{\}}
\end{Highlighting}
\end{Shaded}

\subsection{Setting Up Input}\label{setting-up-input}

\begin{enumerate}
\def\labelenumi{\arabic{enumi}.}
\tightlist
\item
  Create a new Input Action called \textbf{IA\_Collect}
\item
  Open \textbf{IMC\_Default} and add a mapping for \textbf{IA\_Collect}
  with the \textbf{C} key
\item
  In \textbf{BatteryCollectorCharacter.h}, add:
\end{enumerate}

\begin{Shaded}
\begin{Highlighting}[]
\CommentTok{/** Collect Input Action */}
\NormalTok{UPROPERTY}\OperatorTok{(}\NormalTok{EditAnywhere}\OperatorTok{,}\NormalTok{ BlueprintReadOnly}\OperatorTok{,}\NormalTok{ Category }\OperatorTok{=}\NormalTok{ Input}\OperatorTok{,}\NormalTok{ meta }\OperatorTok{=} \OperatorTok{(}\NormalTok{AllowPrivateAccess }\OperatorTok{=} \StringTok{"true"}\OperatorTok{))}
\KeywordTok{class}\NormalTok{ UInputAction}\OperatorTok{*}\NormalTok{ CollectAction}\OperatorTok{;}
\end{Highlighting}
\end{Shaded}

\begin{enumerate}
\def\labelenumi{\arabic{enumi}.}
\setcounter{enumi}{3}
\tightlist
\item
  In \textbf{BatteryCollectorCharacter.cpp}, bind the action:
\end{enumerate}

\begin{Shaded}
\begin{Highlighting}[]
\NormalTok{EnhancedInputComponent}\OperatorTok{{-}\textgreater{}}\NormalTok{BindAction}\OperatorTok{(}\NormalTok{CollectAction}\OperatorTok{,}\NormalTok{ ETriggerEvent}\OperatorTok{::}\NormalTok{Triggered}\OperatorTok{,} \KeywordTok{this}\OperatorTok{,} \OperatorTok{\&}\NormalTok{ABatteryCollectorCharacter}\OperatorTok{::}\NormalTok{CollectPickups}\OperatorTok{);}
\end{Highlighting}
\end{Shaded}

\begin{enumerate}
\def\labelenumi{\arabic{enumi}.}
\setcounter{enumi}{4}
\tightlist
\item
  Set \textbf{IA\_Collect} as the \textbf{Collect Action} in
  \textbf{BP\_ThirdPersonCharacter}
\end{enumerate}

\subsection{Adding WasCollected Function to
Pickup}\label{adding-wascollected-function-to-pickup}

In \textbf{Pickup.h}:

\begin{Shaded}
\begin{Highlighting}[]
\CommentTok{/** Function to call when the pickup is collected */}
\NormalTok{UFUNCTION}\OperatorTok{(}\NormalTok{BlueprintNativeEvent}\OperatorTok{)}
\DataTypeTok{void}\NormalTok{ WasCollected}\OperatorTok{();}
\KeywordTok{virtual} \DataTypeTok{void}\NormalTok{ WasCollected\_Implementation}\OperatorTok{();}
\end{Highlighting}
\end{Shaded}

In \textbf{Pickup.cpp}:

\begin{Shaded}
\begin{Highlighting}[]
\DataTypeTok{void}\NormalTok{ APickup}\OperatorTok{::}\NormalTok{WasCollected\_Implementation}\OperatorTok{()}
\OperatorTok{\{}
    \CommentTok{// Log a debug message}
\NormalTok{    FString PickupDebugString }\OperatorTok{=}\NormalTok{ GetName}\OperatorTok{();}
\NormalTok{    UE\_LOG}\OperatorTok{(}\NormalTok{LogClass}\OperatorTok{,}\NormalTok{ Log}\OperatorTok{,}\NormalTok{ TEXT}\OperatorTok{(}\StringTok{"You have collected }\SpecialCharTok{\%s}\StringTok{"}\OperatorTok{),} \OperatorTok{*}\NormalTok{PickupDebugString}\OperatorTok{);}
\OperatorTok{\}}
\end{Highlighting}
\end{Shaded}

\section{Adding Power to the Game}\label{adding-power-to-the-game}

\subsection{Character Power System}\label{character-power-system}

Add to \textbf{BatteryCollectorCharacter.h}:

\begin{Shaded}
\begin{Highlighting}[]
\KeywordTok{protected}\OperatorTok{:}
    \CommentTok{/** The starting power level of our character */}
\NormalTok{    UPROPERTY}\OperatorTok{(}\NormalTok{EditAnywhere}\OperatorTok{,}\NormalTok{ BlueprintReadWrite}\OperatorTok{,}\NormalTok{ Category }\OperatorTok{=} \StringTok{"Power"}\OperatorTok{,}\NormalTok{ Meta }\OperatorTok{=} \OperatorTok{(}\NormalTok{BlueprintProtected }\OperatorTok{=} \StringTok{"true"}\OperatorTok{))}
    \DataTypeTok{float}\NormalTok{ InitialPower}\OperatorTok{;}
    
    \CommentTok{/** Multiplier for character speed */}
\NormalTok{    UPROPERTY}\OperatorTok{(}\NormalTok{EditAnywhere}\OperatorTok{,}\NormalTok{ BlueprintReadWrite}\OperatorTok{,}\NormalTok{ Category }\OperatorTok{=} \StringTok{"Power"}\OperatorTok{,}\NormalTok{ Meta }\OperatorTok{=} \OperatorTok{(}\NormalTok{BlueprintProtected }\OperatorTok{=} \StringTok{"true"}\OperatorTok{))}
    \DataTypeTok{float}\NormalTok{ SpeedFactor}\OperatorTok{;}
    
    \CommentTok{/** Speed when power level = 0 */}
\NormalTok{    UPROPERTY}\OperatorTok{(}\NormalTok{EditAnywhere}\OperatorTok{,}\NormalTok{ BlueprintReadWrite}\OperatorTok{,}\NormalTok{ Category }\OperatorTok{=} \StringTok{"Power"}\OperatorTok{,}\NormalTok{ Meta }\OperatorTok{=} \OperatorTok{(}\NormalTok{BlueprintProtected }\OperatorTok{=} \StringTok{"true"}\OperatorTok{))}
    \DataTypeTok{float}\NormalTok{ BaseSpeed}\OperatorTok{;}

\KeywordTok{private}\OperatorTok{:}
    \CommentTok{/** Current power level of our character */}
\NormalTok{    UPROPERTY}\OperatorTok{(}\NormalTok{VisibleAnywhere}\OperatorTok{,}\NormalTok{ Category }\OperatorTok{=} \StringTok{"Power"}\OperatorTok{)}
    \DataTypeTok{float}\NormalTok{ CharacterPower}\OperatorTok{;}

\KeywordTok{public}\OperatorTok{:}
    \CommentTok{/** Accessor function for initial power */}
\NormalTok{    UFUNCTION}\OperatorTok{(}\NormalTok{BlueprintPure}\OperatorTok{,}\NormalTok{ Category }\OperatorTok{=} \StringTok{"Power"}\OperatorTok{)}
    \DataTypeTok{float}\NormalTok{ GetInitialPower}\OperatorTok{();}
    
    \CommentTok{/** Accessor function for current power */}
\NormalTok{    UFUNCTION}\OperatorTok{(}\NormalTok{BlueprintPure}\OperatorTok{,}\NormalTok{ Category }\OperatorTok{=} \StringTok{"Power"}\OperatorTok{)}
    \DataTypeTok{float}\NormalTok{ GetCurrentPower}\OperatorTok{();}
    
    \CommentTok{/**}
\CommentTok{    Function to update the character\textquotesingle{}s power }
\CommentTok{    * }\AnnotationTok{@param}\CommentTok{ }\CommentVarTok{PowerChange}\CommentTok{ This is the amount to change the power by, and it can be positive or negative.}
\CommentTok{    */}
\NormalTok{    UFUNCTION}\OperatorTok{(}\NormalTok{BlueprintCallable}\OperatorTok{,}\NormalTok{ Category }\OperatorTok{=} \StringTok{"Power"}\OperatorTok{)}
    \DataTypeTok{void}\NormalTok{ UpdatePower}\OperatorTok{(}\DataTypeTok{float}\NormalTok{ PowerChange}\OperatorTok{);}

\KeywordTok{protected}\OperatorTok{:}
\NormalTok{    UFUNCTION}\OperatorTok{(}\NormalTok{BlueprintImplementableEvent}\OperatorTok{,}\NormalTok{ Category }\OperatorTok{=} \StringTok{"Power"}\OperatorTok{)}
    \DataTypeTok{void}\NormalTok{ PowerChangeEffect}\OperatorTok{();}
\end{Highlighting}
\end{Shaded}

\subsection{Character Power
Implementation}\label{character-power-implementation}

In \textbf{BatteryCollectorCharacter.cpp} constructor:

\begin{Shaded}
\begin{Highlighting}[]
\CommentTok{// Set a base power level for the character}
\NormalTok{InitialPower }\OperatorTok{=} \FloatTok{2000.}\BuiltInTok{f}\OperatorTok{;}
\NormalTok{CharacterPower }\OperatorTok{=}\NormalTok{ InitialPower}\OperatorTok{;}

\CommentTok{// Set the dependence of the speed on the power level}
\NormalTok{SpeedFactor }\OperatorTok{=} \FloatTok{0.75}\BuiltInTok{f}\OperatorTok{;}
\NormalTok{BaseSpeed }\OperatorTok{=} \FloatTok{10.}\BuiltInTok{f}\OperatorTok{;}
\end{Highlighting}
\end{Shaded}

Add the accessor and update functions:

\begin{Shaded}
\begin{Highlighting}[]
\CommentTok{// Reports starting power}
\DataTypeTok{float}\NormalTok{ ABatteryCollectorCharacter}\OperatorTok{::}\NormalTok{GetInitialPower}\OperatorTok{()}
\OperatorTok{\{}
    \ControlFlowTok{return}\NormalTok{ InitialPower}\OperatorTok{;}
\OperatorTok{\}}

\CommentTok{// Reports current power}
\DataTypeTok{float}\NormalTok{ ABatteryCollectorCharacter}\OperatorTok{::}\NormalTok{GetCurrentPower}\OperatorTok{()}
\OperatorTok{\{}
    \ControlFlowTok{return}\NormalTok{ CharacterPower}\OperatorTok{;}
\OperatorTok{\}}

\CommentTok{// Called whenever power is increased or decreased}
\DataTypeTok{void}\NormalTok{ ABatteryCollectorCharacter}\OperatorTok{::}\NormalTok{UpdatePower}\OperatorTok{(}\DataTypeTok{float}\NormalTok{ PowerChange}\OperatorTok{)}
\OperatorTok{\{}
    \CommentTok{// Change power}
\NormalTok{    CharacterPower }\OperatorTok{=}\NormalTok{ CharacterPower }\OperatorTok{+}\NormalTok{ PowerChange}\OperatorTok{;}
    
    \CommentTok{// Change speed based on power}
\NormalTok{    GetCharacterMovement}\OperatorTok{(){-}\textgreater{}}\NormalTok{MaxWalkSpeed }\OperatorTok{=}\NormalTok{ BaseSpeed }\OperatorTok{+}\NormalTok{ SpeedFactor }\OperatorTok{*}\NormalTok{ CharacterPower}\OperatorTok{;}
    
    \CommentTok{// Call visual effect}
\NormalTok{    PowerChangeEffect}\OperatorTok{();}
\OperatorTok{\}}
\end{Highlighting}
\end{Shaded}

\subsection{Powering Up from
Batteries}\label{powering-up-from-batteries}

Update the \textbf{CollectPickups} function:

\begin{Shaded}
\begin{Highlighting}[]
\DataTypeTok{void}\NormalTok{ ABatteryCollectorCharacter}\OperatorTok{::}\NormalTok{CollectPickups}\OperatorTok{()}
\OperatorTok{\{}
    \CommentTok{// Get all overlapping Actors and store them in an array}
\NormalTok{    TArray}\OperatorTok{\textless{}}\NormalTok{AActor}\OperatorTok{*\textgreater{}}\NormalTok{ CollectedActors}\OperatorTok{;}
\NormalTok{    CollectionSphere}\OperatorTok{{-}\textgreater{}}\NormalTok{GetOverlappingActors}\OperatorTok{(}\NormalTok{CollectedActors}\OperatorTok{);}
    
    \CommentTok{// Keep track of the collected power}
    \DataTypeTok{float}\NormalTok{ CollectedPower }\OperatorTok{=} \DecValTok{0}\OperatorTok{;}
    
    \CommentTok{// For each actor we collected}
    \ControlFlowTok{for} \OperatorTok{(}\NormalTok{int32 iCollected }\OperatorTok{=} \DecValTok{0}\OperatorTok{;}\NormalTok{ iCollected }\OperatorTok{\textless{}}\NormalTok{ CollectedActors}\OperatorTok{.}\NormalTok{Num}\OperatorTok{();} \OperatorTok{++}\NormalTok{iCollected}\OperatorTok{)}
    \OperatorTok{\{}
        \CommentTok{// Cast the actor to APickup}
\NormalTok{        APickup}\OperatorTok{*} \AttributeTok{const}\NormalTok{ TestPickup }\OperatorTok{=}\NormalTok{ Cast}\OperatorTok{\textless{}}\NormalTok{APickup}\OperatorTok{\textgreater{}(}\NormalTok{CollectedActors}\OperatorTok{[}\NormalTok{iCollected}\OperatorTok{]);}
        
        \CommentTok{// If the cast is successful and the pickup is valid and active}
        \ControlFlowTok{if} \OperatorTok{(}\NormalTok{TestPickup }\OperatorTok{\&\&}\NormalTok{ IsValid}\OperatorTok{(}\NormalTok{TestPickup}\OperatorTok{)} \OperatorTok{\&\&}\NormalTok{ TestPickup}\OperatorTok{{-}\textgreater{}}\NormalTok{IsActive}\OperatorTok{())}
        \OperatorTok{\{}
            \CommentTok{// Call the pickup\textquotesingle{}s WasCollected function}
\NormalTok{            TestPickup}\OperatorTok{{-}\textgreater{}}\NormalTok{WasCollected}\OperatorTok{();}
            
            \CommentTok{// Check to see if the pickup is also a battery}
\NormalTok{            ABatteryPickup}\OperatorTok{*} \AttributeTok{const}\NormalTok{ TestBattery }\OperatorTok{=}\NormalTok{ Cast}\OperatorTok{\textless{}}\NormalTok{ABatteryPickup}\OperatorTok{\textgreater{}(}\NormalTok{TestPickup}\OperatorTok{);}
            \ControlFlowTok{if} \OperatorTok{(}\NormalTok{TestBattery}\OperatorTok{)}
            \OperatorTok{\{}
                \CommentTok{// Increase the collected power}
\NormalTok{                CollectedPower }\OperatorTok{+=}\NormalTok{ TestBattery}\OperatorTok{{-}\textgreater{}}\NormalTok{GetPower}\OperatorTok{();}
            \OperatorTok{\}}
            
            \CommentTok{// Deactivate the pickup}
\NormalTok{            TestPickup}\OperatorTok{{-}\textgreater{}}\NormalTok{SetActive}\OperatorTok{(}\KeywordTok{false}\OperatorTok{);}
        \OperatorTok{\}}
    \OperatorTok{\}}
    
    \ControlFlowTok{if} \OperatorTok{(}\NormalTok{CollectedPower }\OperatorTok{\textgreater{}} \DecValTok{0}\OperatorTok{)}
    \OperatorTok{\{}
\NormalTok{        UpdatePower}\OperatorTok{(}\NormalTok{CollectedPower}\OperatorTok{);}
    \OperatorTok{\}}
\OperatorTok{\}}
\end{Highlighting}
\end{Shaded}

\section{GameMode Power Drain}\label{gamemode-power-drain}

\subsection{GameMode Power System}\label{gamemode-power-system}

Add to \textbf{BatteryCollectorGameMode.h}:

\begin{Shaded}
\begin{Highlighting}[]
\KeywordTok{protected}\OperatorTok{:}
    \CommentTok{/** The rate at which the character loses power */}
\NormalTok{    UPROPERTY}\OperatorTok{(}\NormalTok{EditDefaultsOnly}\OperatorTok{,}\NormalTok{ BlueprintReadWrite}\OperatorTok{,}\NormalTok{ Category }\OperatorTok{=} \StringTok{"Power"}\OperatorTok{,}\NormalTok{ Meta }\OperatorTok{=} \OperatorTok{(}\NormalTok{BlueprintProtected }\OperatorTok{=} \KeywordTok{true}\OperatorTok{))}
    \DataTypeTok{float}\NormalTok{ DecayRate}\OperatorTok{;}
    
    \CommentTok{/** The power needed to win the game */}
\NormalTok{    UPROPERTY}\OperatorTok{(}\NormalTok{EditDefaultsOnly}\OperatorTok{,}\NormalTok{ BlueprintReadWrite}\OperatorTok{,}\NormalTok{ Category }\OperatorTok{=} \StringTok{"Power"}\OperatorTok{,}\NormalTok{ Meta }\OperatorTok{=} \OperatorTok{(}\NormalTok{BlueprintProtected }\OperatorTok{=} \KeywordTok{true}\OperatorTok{))}
    \DataTypeTok{float}\NormalTok{ PowerToWin}\OperatorTok{;}

\KeywordTok{public}\OperatorTok{:}
    \KeywordTok{virtual} \DataTypeTok{void}\NormalTok{ Tick}\OperatorTok{(}\DataTypeTok{float}\NormalTok{ DeltaTime}\OperatorTok{)} \KeywordTok{override}\OperatorTok{;}
    \KeywordTok{virtual} \DataTypeTok{void}\NormalTok{ BeginPlay}\OperatorTok{()} \KeywordTok{override}\OperatorTok{;}
    
    \CommentTok{/** Returns the power needed to win {-} needed for the HUD */}
\NormalTok{    UFUNCTION}\OperatorTok{(}\NormalTok{BlueprintPure}\OperatorTok{,}\NormalTok{ Category }\OperatorTok{=} \StringTok{"Power"}\OperatorTok{)}
    \DataTypeTok{float}\NormalTok{ GetPowerToWin}\OperatorTok{()} \AttributeTok{const}\OperatorTok{;}
\end{Highlighting}
\end{Shaded}

\subsection{GameMode Implementation}\label{gamemode-implementation}

In \textbf{BatteryCollectorGameMode.cpp} constructor:

\begin{Shaded}
\begin{Highlighting}[]
\CommentTok{// Base decay rate}
\NormalTok{DecayRate }\OperatorTok{=} \FloatTok{0.01}\BuiltInTok{f}\OperatorTok{;}

\NormalTok{PrimaryActorTick}\OperatorTok{.}\NormalTok{bCanEverTick }\OperatorTok{=} \KeywordTok{true}\OperatorTok{;}
\end{Highlighting}
\end{Shaded}

Add the implementation:

\begin{Shaded}
\begin{Highlighting}[]
\PreprocessorTok{\#include }\ImportTok{\textless{}Kismet/GameplayStatics.h\textgreater{}}

\DataTypeTok{void}\NormalTok{ ABatteryCollectorGameMode}\OperatorTok{::}\NormalTok{BeginPlay}\OperatorTok{()}
\OperatorTok{\{}
\NormalTok{    Super}\OperatorTok{::}\NormalTok{BeginPlay}\OperatorTok{();}
    
    \CommentTok{// Set score to beat}
\NormalTok{    ABatteryCollectorCharacter}\OperatorTok{*}\NormalTok{ MyCharacter }\OperatorTok{=}\NormalTok{ Cast}\OperatorTok{\textless{}}\NormalTok{ABatteryCollectorCharacter}\OperatorTok{\textgreater{}(}\NormalTok{UGameplayStatics}\OperatorTok{::}\NormalTok{GetPlayerPawn}\OperatorTok{(}\KeywordTok{this}\OperatorTok{,} \DecValTok{0}\OperatorTok{));}
    \ControlFlowTok{if} \OperatorTok{(}\NormalTok{MyCharacter}\OperatorTok{)}
    \OperatorTok{\{}
\NormalTok{        PowerToWin }\OperatorTok{=} \OperatorTok{(}\NormalTok{MyCharacter}\OperatorTok{{-}\textgreater{}}\NormalTok{GetInitialPower}\OperatorTok{())} \OperatorTok{*} \FloatTok{1.25}\BuiltInTok{f}\OperatorTok{;}
    \OperatorTok{\}}
\OperatorTok{\}}

\DataTypeTok{void}\NormalTok{ ABatteryCollectorGameMode}\OperatorTok{::}\NormalTok{Tick}\OperatorTok{(}\DataTypeTok{float}\NormalTok{ DeltaTime}\OperatorTok{)}
\OperatorTok{\{}
\NormalTok{    Super}\OperatorTok{::}\NormalTok{Tick}\OperatorTok{(}\NormalTok{DeltaTime}\OperatorTok{);}
    
    \CommentTok{// Check that we are using the battery collector character }
\NormalTok{    ABatteryCollectorCharacter}\OperatorTok{*}\NormalTok{ MyCharacter }\OperatorTok{=}\NormalTok{ Cast}\OperatorTok{\textless{}}\NormalTok{ABatteryCollectorCharacter}\OperatorTok{\textgreater{}(}\NormalTok{UGameplayStatics}\OperatorTok{::}\NormalTok{GetPlayerPawn}\OperatorTok{(}\KeywordTok{this}\OperatorTok{,} \DecValTok{0}\OperatorTok{));}
    \ControlFlowTok{if} \OperatorTok{(}\NormalTok{MyCharacter}\OperatorTok{)}
    \OperatorTok{\{}
        \CommentTok{// If the character\textquotesingle{}s power is positive}
        \ControlFlowTok{if} \OperatorTok{(}\NormalTok{MyCharacter}\OperatorTok{{-}\textgreater{}}\NormalTok{GetCurrentPower}\OperatorTok{()} \OperatorTok{\textgreater{}} \DecValTok{0}\OperatorTok{)}
        \OperatorTok{\{}
            \CommentTok{// Decrease the character\textquotesingle{}s power using the decay rate}
\NormalTok{            MyCharacter}\OperatorTok{{-}\textgreater{}}\NormalTok{UpdatePower}\OperatorTok{({-}}\NormalTok{DeltaTime }\OperatorTok{*}\NormalTok{ DecayRate }\OperatorTok{*} \OperatorTok{(}\NormalTok{MyCharacter}\OperatorTok{{-}\textgreater{}}\NormalTok{GetInitialPower}\OperatorTok{()));}
        \OperatorTok{\}}
    \OperatorTok{\}}
\OperatorTok{\}}

\DataTypeTok{float}\NormalTok{ ABatteryCollectorGameMode}\OperatorTok{::}\NormalTok{GetPowerToWin}\OperatorTok{()} \AttributeTok{const}
\OperatorTok{\{}
    \ControlFlowTok{return}\NormalTok{ PowerToWin}\OperatorTok{;}
\OperatorTok{\}}
\end{Highlighting}
\end{Shaded}

\section{Character Visual Effects}\label{character-visual-effects}

\subsection{Material Color Changes}\label{material-color-changes}

In \textbf{BP\_ThirdPersonCharacter}, set up the
\textbf{PowerChangeEffect} Blueprint event:

\begin{enumerate}
\def\labelenumi{\arabic{enumi}.}
\tightlist
\item
  Create a \textbf{Dynamic Material Instance} in the Construction Script
\item
  In the Event Graph, create an \textbf{Event PowerChangeEffect}
\item
  Set up a \textbf{Lerp} between two colors based on Current Power /
  Initial Power ratio
\item
  Use \textbf{Set Vector Parameter Value} to change the material tint
\end{enumerate}

\subsection{Speed Changes}\label{speed-changes}

The speed changes are already implemented in the \textbf{UpdatePower}
function in C++, where \textbf{MaxWalkSpeed} is updated based on the
character's power level.

\section{Creating Particle Effects}\label{creating-particle-effects}

\subsection{Electric Arc Material}\label{electric-arc-material}

\begin{enumerate}
\def\labelenumi{\arabic{enumi}.}
\tightlist
\item
  Create a new \textbf{Material} called \textbf{M\_ElectricArc}
\item
  Set \textbf{Shading Model} to \textbf{Unlit}
\item
  Connect \textbf{Particle Color} to \textbf{Emissive Color}
\end{enumerate}

\subsection{Niagara Particle System}\label{niagara-particle-system}

\begin{enumerate}
\def\labelenumi{\arabic{enumi}.}
\tightlist
\item
  Create a new \textbf{Niagara System} called \textbf{P\_ElectricArcs}
\item
  Use \textbf{Static Beam} emitter
\item
  Set up \textbf{BeamEnd} as a user-exposed parameter
\item
  Configure material and beam properties
\end{enumerate}

\subsection{Adding Particles to
Battery}\label{adding-particles-to-battery}

In \textbf{Battery\_BP}, add the particle effect to the
\textbf{WasCollected} event:

\begin{enumerate}
\def\labelenumi{\arabic{enumi}.}
\tightlist
\item
  \textbf{Spawn System Attached} with \textbf{P\_ElectricArcs}
\item
  Get the player character's socket location
\item
  Set the \textbf{BeamEnd} parameter to target the character
\item
  Add delay and parent function call
\end{enumerate}

\section{HUD System}\label{hud-system}

\subsection{Enabling UMG}\label{enabling-umg}

Add to \textbf{BatteryCollector.Build.cs}:

\begin{Shaded}
\begin{Highlighting}[]
\NormalTok{PublicDependencyModuleNames}\OperatorTok{.}\FunctionTok{AddRange}\OperatorTok{(}\KeywordTok{new} \DataTypeTok{string}\OperatorTok{[]} \OperatorTok{\{} \StringTok{"Core"}\OperatorTok{,} \StringTok{"CoreUObject"}\OperatorTok{,} \StringTok{"Engine"}\OperatorTok{,} \StringTok{"InputCore"}\OperatorTok{,} \StringTok{"HeadMountedDisplay"}\OperatorTok{,} \StringTok{"UMG"}\OperatorTok{\});}

\NormalTok{PrivateDependencyModuleNames}\OperatorTok{.}\FunctionTok{AddRange}\OperatorTok{(}\KeywordTok{new} \DataTypeTok{string}\OperatorTok{[]} \OperatorTok{\{} \StringTok{"Slate"}\OperatorTok{,} \StringTok{"SlateCore"} \OperatorTok{\});}
\end{Highlighting}
\end{Shaded}

\subsection{GameMode HUD Setup}\label{gamemode-hud-setup}

Add to \textbf{BatteryCollectorGameMode.h}:

\begin{Shaded}
\begin{Highlighting}[]
\PreprocessorTok{\#include }\ImportTok{\textless{}UserWidget.h\textgreater{}}

\KeywordTok{protected}\OperatorTok{:}
    \CommentTok{/** The Widget class to use for our HUD screen */}
\NormalTok{    UPROPERTY}\OperatorTok{(}\NormalTok{EditDefaultsOnly}\OperatorTok{,}\NormalTok{ BlueprintReadWrite}\OperatorTok{,}\NormalTok{ Category }\OperatorTok{=} \StringTok{"Power"}\OperatorTok{,}\NormalTok{ Meta }\OperatorTok{=} \OperatorTok{(}\NormalTok{BlueprintProtected }\OperatorTok{=} \KeywordTok{true}\OperatorTok{))}
\NormalTok{    TSubclassOf}\OperatorTok{\textless{}}\KeywordTok{class}\NormalTok{ UUserWidget}\OperatorTok{\textgreater{}}\NormalTok{ HUDWidgetClass}\OperatorTok{;}
    
    \CommentTok{/** The instance of the HUD */}
\NormalTok{    UPROPERTY}\OperatorTok{()}
    \KeywordTok{class}\NormalTok{ UUserWidget}\OperatorTok{*}\NormalTok{ CurrentWidget}\OperatorTok{;}
\end{Highlighting}
\end{Shaded}

In \textbf{BeginPlay}, add:

\begin{Shaded}
\begin{Highlighting}[]
\ControlFlowTok{if} \OperatorTok{(}\NormalTok{HUDWidgetClass }\OperatorTok{!=} \KeywordTok{nullptr}\OperatorTok{)}
\OperatorTok{\{}
\NormalTok{    CurrentWidget }\OperatorTok{=}\NormalTok{ CreateWidget}\OperatorTok{\textless{}}\NormalTok{UUserWidget}\OperatorTok{\textgreater{}(}\NormalTok{GetWorld}\OperatorTok{(),}\NormalTok{ HUDWidgetClass}\OperatorTok{);}
    \ControlFlowTok{if} \OperatorTok{(}\NormalTok{CurrentWidget }\OperatorTok{!=} \KeywordTok{nullptr}\OperatorTok{)}
    \OperatorTok{\{}
\NormalTok{        CurrentWidget}\OperatorTok{{-}\textgreater{}}\NormalTok{AddToViewport}\OperatorTok{();}
    \OperatorTok{\}}
\OperatorTok{\}}
\end{Highlighting}
\end{Shaded}

\subsection{HUD Blueprint}\label{hud-blueprint}

Create \textbf{BatteryHUD} Widget Blueprint with:

\begin{enumerate}
\def\labelenumi{\arabic{enumi}.}
\tightlist
\item
  \textbf{Progress Bar} for power level
\item
  \textbf{Text Block} for game state messages
\item
  Bind progress bar to \textbf{Current Power / Power To Win}
\item
  Bind text to game state messages
\end{enumerate}

\section{Game States}\label{game-states}

\subsection{Play State Enum}\label{play-state-enum}

Add to \textbf{BatteryCollectorGameMode.h}:

\begin{Shaded}
\begin{Highlighting}[]
\CommentTok{// Enum to store the current state of gameplay}
\NormalTok{UENUM}\OperatorTok{(}\NormalTok{BlueprintType}\OperatorTok{)}
\KeywordTok{enum} \KeywordTok{class}\NormalTok{ EBatteryPlayState }\OperatorTok{:}\NormalTok{ uint8}
\OperatorTok{\{}
\NormalTok{    EPlaying}\OperatorTok{,}
\NormalTok{    EGameOver}\OperatorTok{,}
\NormalTok{    EWon}\OperatorTok{,}
\NormalTok{    EUnknown}
\OperatorTok{\};}

\KeywordTok{private}\OperatorTok{:}
    \CommentTok{/** Keeps track of the current playing state */}
\NormalTok{    EBatteryPlayState CurrentState}\OperatorTok{;}
    
\NormalTok{    TArray}\OperatorTok{\textless{}}\KeywordTok{class}\NormalTok{ ASpawnVolume}\OperatorTok{*\textgreater{}}\NormalTok{ SpawnVolumeActors}\OperatorTok{;}
    
    \CommentTok{/** Handle any function calls that rely upon changing the playing state of our game */}
    \DataTypeTok{void}\NormalTok{ HandleNewState}\OperatorTok{(}\NormalTok{EBatteryPlayState NewState}\OperatorTok{);}

\KeywordTok{public}\OperatorTok{:}
    \CommentTok{/** Returns the current playing state */}
\NormalTok{    UFUNCTION}\OperatorTok{(}\NormalTok{BlueprintPure}\OperatorTok{,}\NormalTok{ Category }\OperatorTok{=} \StringTok{"Power"}\OperatorTok{)}
\NormalTok{    EBatteryPlayState GetCurrentState}\OperatorTok{()} \AttributeTok{const}\OperatorTok{;}
    
    \CommentTok{/** Sets a new playing state */}
    \DataTypeTok{void}\NormalTok{ SetCurrentState}\OperatorTok{(}\NormalTok{EBatteryPlayState NewState}\OperatorTok{);}
\end{Highlighting}
\end{Shaded}

\subsection{State Management
Implementation}\label{state-management-implementation}

```cpp \#include ``SpawnVolume.h''

void ABatteryCollectorGameMode::BeginPlay() \{ Super::BeginPlay();

\begin{verbatim}
// Find all spawn volume Actors
TArray<AActor*> FoundActors;
UGameplayStatics::GetAllActorsOfClass(GetWorld(), ASpawnVolume::StaticClass(), FoundActors);

for (auto Actor : FoundActors)
{
    ASpawnVolume* SpawnVolumeActor = Cast<ASpawnVolume>(Actor);
    if (SpawnVolumeActor)
    {
        SpawnVolumeActors.AddUnique(SpawnVolumeActor);
    }
}

SetCurrentState(EBatteryPlayState::EPlaying);

// Set score to beat
ABatteryCollectorCharacter* MyCharacter = Cast<ABatteryCollectorCharacter>(UGameplayStatics::GetPlayerPawn(this, 0));
if (MyCharacter)
{
    PowerToWin = (MyCharacter->GetInitialPower()) * 1.25f;
}

if (HUDWidgetClass != nullptr)
{
    CurrentWidget = CreateWidget<UUserWidget>(GetWorld(), HUDWidgetClass);
    if (CurrentWidget != nullptr)
    {
        CurrentWidget->AddToViewport();
    }
}
\end{verbatim}

\}

EBatteryPlayState ABatteryCollectorGameMode::GetCurrentState() const \{
return CurrentState; \}

void ABatteryCollectorGameMode::SetCurrentState(EBatteryPlayState
NewState) \{ // Set current state CurrentState = NewState; // Handle the
new current state HandleNewState(CurrentState); \}

void ABatteryCollectorGameMode::HandleNewState(EBatteryPlayState
NewState) \{ switch (NewState) \{ // If the game is playing case
EBatteryPlayState::EPlaying: \{ // Spawn volumes active for
(ASpawnVolume* Volume : SpawnVolumeActors) \{
Volume-\textgreater SetSpawningActive(true); \} \} break;

\begin{verbatim}
    // If we've won the game 
    case EBatteryPlayState::EWon:
    {
        // Spawn volumes inactive
        for (ASpawnVolume* Volume : SpawnVolumeActors)
        {
            Volume->SetSpawningActive(false);
        }
    }
    break;
    
    // If we've lost the game
    case EBatteryPlayState::EGameOver:
    {
        // Spawn volumes inactive
        for (ASpawnVolume* Volume : SpawnVolumeActors)
        {
            Volume->SetSpawningActive(false);
        }
        
        // Block player input
        APlayerController* PlayerController = UGameplayStatics::GetPlayerController(this, 0);
        if (PlayerController)
        {
            PlayerController->SetCinematicMode(true, false, false, true, true);
        }
        
        // Ragdoll the character
        ACharacter* MyCharacter = UGameplayStatics::GetPlayerCharacter(this, 0);
        if (MyCharacter)
        {
            MyCharacter->GetMesh()->SetSimulatePhysics(true);
            MyCharacter->GetMovementComponent()->MovementState.bCanJump = false;
        }
    }
    break;
    
    // Unknown/default state
    default:
    case EBatteryPlayState::EUnknown:
    {
        // Do nothing
    }
    break;
}
\end{verbatim}

\}




\end{document}
