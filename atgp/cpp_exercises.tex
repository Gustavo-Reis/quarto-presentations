% Options for packages loaded elsewhere
% Options for packages loaded elsewhere
\PassOptionsToPackage{unicode}{hyperref}
\PassOptionsToPackage{hyphens}{url}
\PassOptionsToPackage{dvipsnames,svgnames,x11names}{xcolor}
%
\documentclass[
]{article}
\usepackage{xcolor}
\usepackage[margin=1in]{geometry}
\usepackage{amsmath,amssymb}
\setcounter{secnumdepth}{-\maxdimen} % remove section numbering
\usepackage{iftex}
\ifPDFTeX
  \usepackage[T1]{fontenc}
  \usepackage[utf8]{inputenc}
  \usepackage{textcomp} % provide euro and other symbols
\else % if luatex or xetex
  \usepackage{unicode-math} % this also loads fontspec
  \defaultfontfeatures{Scale=MatchLowercase}
  \defaultfontfeatures[\rmfamily]{Ligatures=TeX,Scale=1}
\fi
\usepackage{lmodern}
\ifPDFTeX\else
  % xetex/luatex font selection
\fi
% Use upquote if available, for straight quotes in verbatim environments
\IfFileExists{upquote.sty}{\usepackage{upquote}}{}
\IfFileExists{microtype.sty}{% use microtype if available
  \usepackage[]{microtype}
  \UseMicrotypeSet[protrusion]{basicmath} % disable protrusion for tt fonts
}{}
\makeatletter
\@ifundefined{KOMAClassName}{% if non-KOMA class
  \IfFileExists{parskip.sty}{%
    \usepackage{parskip}
  }{% else
    \setlength{\parindent}{0pt}
    \setlength{\parskip}{6pt plus 2pt minus 1pt}}
}{% if KOMA class
  \KOMAoptions{parskip=half}}
\makeatother
% Make \paragraph and \subparagraph free-standing
\makeatletter
\ifx\paragraph\undefined\else
  \let\oldparagraph\paragraph
  \renewcommand{\paragraph}{
    \@ifstar
      \xxxParagraphStar
      \xxxParagraphNoStar
  }
  \newcommand{\xxxParagraphStar}[1]{\oldparagraph*{#1}\mbox{}}
  \newcommand{\xxxParagraphNoStar}[1]{\oldparagraph{#1}\mbox{}}
\fi
\ifx\subparagraph\undefined\else
  \let\oldsubparagraph\subparagraph
  \renewcommand{\subparagraph}{
    \@ifstar
      \xxxSubParagraphStar
      \xxxSubParagraphNoStar
  }
  \newcommand{\xxxSubParagraphStar}[1]{\oldsubparagraph*{#1}\mbox{}}
  \newcommand{\xxxSubParagraphNoStar}[1]{\oldsubparagraph{#1}\mbox{}}
\fi
\makeatother

\usepackage{color}
\usepackage{fancyvrb}
\newcommand{\VerbBar}{|}
\newcommand{\VERB}{\Verb[commandchars=\\\{\}]}
\DefineVerbatimEnvironment{Highlighting}{Verbatim}{commandchars=\\\{\}}
% Add ',fontsize=\small' for more characters per line
\usepackage{framed}
\definecolor{shadecolor}{RGB}{241,243,245}
\newenvironment{Shaded}{\begin{snugshade}}{\end{snugshade}}
\newcommand{\AlertTok}[1]{\textcolor[rgb]{0.68,0.00,0.00}{#1}}
\newcommand{\AnnotationTok}[1]{\textcolor[rgb]{0.37,0.37,0.37}{#1}}
\newcommand{\AttributeTok}[1]{\textcolor[rgb]{0.40,0.45,0.13}{#1}}
\newcommand{\BaseNTok}[1]{\textcolor[rgb]{0.68,0.00,0.00}{#1}}
\newcommand{\BuiltInTok}[1]{\textcolor[rgb]{0.00,0.23,0.31}{#1}}
\newcommand{\CharTok}[1]{\textcolor[rgb]{0.13,0.47,0.30}{#1}}
\newcommand{\CommentTok}[1]{\textcolor[rgb]{0.37,0.37,0.37}{#1}}
\newcommand{\CommentVarTok}[1]{\textcolor[rgb]{0.37,0.37,0.37}{\textit{#1}}}
\newcommand{\ConstantTok}[1]{\textcolor[rgb]{0.56,0.35,0.01}{#1}}
\newcommand{\ControlFlowTok}[1]{\textcolor[rgb]{0.00,0.23,0.31}{\textbf{#1}}}
\newcommand{\DataTypeTok}[1]{\textcolor[rgb]{0.68,0.00,0.00}{#1}}
\newcommand{\DecValTok}[1]{\textcolor[rgb]{0.68,0.00,0.00}{#1}}
\newcommand{\DocumentationTok}[1]{\textcolor[rgb]{0.37,0.37,0.37}{\textit{#1}}}
\newcommand{\ErrorTok}[1]{\textcolor[rgb]{0.68,0.00,0.00}{#1}}
\newcommand{\ExtensionTok}[1]{\textcolor[rgb]{0.00,0.23,0.31}{#1}}
\newcommand{\FloatTok}[1]{\textcolor[rgb]{0.68,0.00,0.00}{#1}}
\newcommand{\FunctionTok}[1]{\textcolor[rgb]{0.28,0.35,0.67}{#1}}
\newcommand{\ImportTok}[1]{\textcolor[rgb]{0.00,0.46,0.62}{#1}}
\newcommand{\InformationTok}[1]{\textcolor[rgb]{0.37,0.37,0.37}{#1}}
\newcommand{\KeywordTok}[1]{\textcolor[rgb]{0.00,0.23,0.31}{\textbf{#1}}}
\newcommand{\NormalTok}[1]{\textcolor[rgb]{0.00,0.23,0.31}{#1}}
\newcommand{\OperatorTok}[1]{\textcolor[rgb]{0.37,0.37,0.37}{#1}}
\newcommand{\OtherTok}[1]{\textcolor[rgb]{0.00,0.23,0.31}{#1}}
\newcommand{\PreprocessorTok}[1]{\textcolor[rgb]{0.68,0.00,0.00}{#1}}
\newcommand{\RegionMarkerTok}[1]{\textcolor[rgb]{0.00,0.23,0.31}{#1}}
\newcommand{\SpecialCharTok}[1]{\textcolor[rgb]{0.37,0.37,0.37}{#1}}
\newcommand{\SpecialStringTok}[1]{\textcolor[rgb]{0.13,0.47,0.30}{#1}}
\newcommand{\StringTok}[1]{\textcolor[rgb]{0.13,0.47,0.30}{#1}}
\newcommand{\VariableTok}[1]{\textcolor[rgb]{0.07,0.07,0.07}{#1}}
\newcommand{\VerbatimStringTok}[1]{\textcolor[rgb]{0.13,0.47,0.30}{#1}}
\newcommand{\WarningTok}[1]{\textcolor[rgb]{0.37,0.37,0.37}{\textit{#1}}}

\usepackage{longtable,booktabs,array}
\usepackage{calc} % for calculating minipage widths
% Correct order of tables after \paragraph or \subparagraph
\usepackage{etoolbox}
\makeatletter
\patchcmd\longtable{\par}{\if@noskipsec\mbox{}\fi\par}{}{}
\makeatother
% Allow footnotes in longtable head/foot
\IfFileExists{footnotehyper.sty}{\usepackage{footnotehyper}}{\usepackage{footnote}}
\makesavenoteenv{longtable}
\usepackage{graphicx}
\makeatletter
\newsavebox\pandoc@box
\newcommand*\pandocbounded[1]{% scales image to fit in text height/width
  \sbox\pandoc@box{#1}%
  \Gscale@div\@tempa{\textheight}{\dimexpr\ht\pandoc@box+\dp\pandoc@box\relax}%
  \Gscale@div\@tempb{\linewidth}{\wd\pandoc@box}%
  \ifdim\@tempb\p@<\@tempa\p@\let\@tempa\@tempb\fi% select the smaller of both
  \ifdim\@tempa\p@<\p@\scalebox{\@tempa}{\usebox\pandoc@box}%
  \else\usebox{\pandoc@box}%
  \fi%
}
% Set default figure placement to htbp
\def\fps@figure{htbp}
\makeatother





\setlength{\emergencystretch}{3em} % prevent overfull lines

\providecommand{\tightlist}{%
  \setlength{\itemsep}{0pt}\setlength{\parskip}{0pt}}



 


\usepackage{fancyhdr}
\pagestyle{fancy}
\fancyhead[L]{Advanced Game Programming Topics}
\fancyhead[R]{IPL - ESTG}
\makeatletter
\@ifpackageloaded{caption}{}{\usepackage{caption}}
\AtBeginDocument{%
\ifdefined\contentsname
  \renewcommand*\contentsname{Table of contents}
\else
  \newcommand\contentsname{Table of contents}
\fi
\ifdefined\listfigurename
  \renewcommand*\listfigurename{List of Figures}
\else
  \newcommand\listfigurename{List of Figures}
\fi
\ifdefined\listtablename
  \renewcommand*\listtablename{List of Tables}
\else
  \newcommand\listtablename{List of Tables}
\fi
\ifdefined\figurename
  \renewcommand*\figurename{Figure}
\else
  \newcommand\figurename{Figure}
\fi
\ifdefined\tablename
  \renewcommand*\tablename{Table}
\else
  \newcommand\tablename{Table}
\fi
}
\@ifpackageloaded{float}{}{\usepackage{float}}
\floatstyle{ruled}
\@ifundefined{c@chapter}{\newfloat{codelisting}{h}{lop}}{\newfloat{codelisting}{h}{lop}[chapter]}
\floatname{codelisting}{Listing}
\newcommand*\listoflistings{\listof{codelisting}{List of Listings}}
\makeatother
\makeatletter
\makeatother
\makeatletter
\@ifpackageloaded{caption}{}{\usepackage{caption}}
\@ifpackageloaded{subcaption}{}{\usepackage{subcaption}}
\makeatother
\usepackage{bookmark}
\IfFileExists{xurl.sty}{\usepackage{xurl}}{} % add URL line breaks if available
\urlstyle{same}
\hypersetup{
  pdftitle={Modularity Exercise: Space Shooter Game},
  pdfauthor={Departamento de Engenharia Informática},
  colorlinks=true,
  linkcolor={blue},
  filecolor={Maroon},
  citecolor={Blue},
  urlcolor={Blue},
  pdfcreator={LaTeX via pandoc}}


\title{Modularity Exercise: Space Shooter Game}
\usepackage{etoolbox}
\makeatletter
\providecommand{\subtitle}[1]{% add subtitle to \maketitle
  \apptocmd{\@title}{\par {\large #1 \par}}{}{}
}
\makeatother
\subtitle{Advanced Game Programming Topics}
\author{Departamento de Engenharia Informática}
\date{2025-10-10}
\begin{document}
\maketitle


\section{Problem Description}\label{problem-description}

Consider a video game like \emph{Xenon}, where, among other objects,
there exists a spaceship that fires shots against enemies. In this
assignment, you will model and implement classes corresponding to the
\textbf{Spaceship}, \textbf{Gun}, and \textbf{Shot} objects using modern
C++ best practices.

This exercise focuses on:

\begin{itemize}
\tightlist
\item
  \textbf{Modularity}: Proper separation of interface and implementation
\item
  \textbf{RAII}: Resource Acquisition Is Initialization
\item
  \textbf{Modern C++}: Using C++17/20 features
\item
  \textbf{Class Design}: Concrete classes with proper encapsulation
\end{itemize}

\section{Object Requirements}\label{object-requirements}

\subsection{1. Position2D Structure}\label{position2d-structure}

Create a simple structure to represent 2D positions:

\begin{Shaded}
\begin{Highlighting}[]
\KeywordTok{struct}\NormalTok{ Position2D }\OperatorTok{\{}
    \DataTypeTok{double}\NormalTok{ x}\OperatorTok{;}
    \DataTypeTok{double}\NormalTok{ y}\OperatorTok{;}
\OperatorTok{\};}
\end{Highlighting}
\end{Shaded}

\subsection{2. Vector2D Structure}\label{vector2d-structure}

Create a structure to represent 2D velocity vectors:

\begin{Shaded}
\begin{Highlighting}[]
\KeywordTok{struct}\NormalTok{ Vector2D }\OperatorTok{\{}
    \DataTypeTok{double}\NormalTok{ x}\OperatorTok{;}
    \DataTypeTok{double}\NormalTok{ y}\OperatorTok{;}
\OperatorTok{\};}
\end{Highlighting}
\end{Shaded}

\subsection{3. Shot Class}\label{shot-class}

A \textbf{Shot} represents a projectile fired by the gun.

\subsubsection{Attributes:}\label{attributes}

\begin{itemize}
\tightlist
\item
  Position (x, y coordinates)
\item
  Velocity vector (speed and direction)
\item
  Destruction power (damage value)
\item
  Active status (whether the shot is still active)
\end{itemize}

\subsubsection{Behaviors:}\label{behaviors}

\begin{itemize}
\tightlist
\item
  Move the shot based on its velocity
\item
  Check if the shot is still within bounds
\item
  Get/set shot properties
\item
  Deactivate the shot
\end{itemize}

\subsubsection{Requirements:}\label{requirements}

\begin{itemize}
\tightlist
\item
  Use proper encapsulation (private data members)
\item
  Provide const-correct member functions
\item
  Implement a method to update position:
  \texttt{void\ update(double\ deltaTime)}
\item
  Implement bounds checking:
  \texttt{bool\ isInBounds(double\ maxX,\ double\ maxY)\ const}
\end{itemize}

\subsection{4. Gun Class}\label{gun-class}

A \textbf{Gun} manages shooting mechanics and cooldown.

\subsubsection{Attributes:}\label{attributes-1}

\begin{itemize}
\tightlist
\item
  Cooldown time between shots (in seconds)
\item
  Current cooldown remaining
\item
  Maximum ammunition capacity
\item
  Current ammunition count
\item
  Shot template (power, speed)
\end{itemize}

\subsubsection{Behaviors:}\label{behaviors-1}

\begin{itemize}
\tightlist
\item
  Fire a shot (if cooldown allows and ammunition available)
\item
  Update cooldown timer
\item
  Reload ammunition
\item
  Check if ready to fire
\end{itemize}

\subsubsection{Requirements:}\label{requirements-1}

\begin{itemize}
\tightlist
\item
  Return \texttt{std::optional\textless{}Shot\textgreater{}} from
  \texttt{fire()} method (C++17 feature)
\item
  Implement cooldown management:
  \texttt{void\ update(double\ deltaTime)}
\item
  Implement \texttt{bool\ canFire()\ const} to check firing readiness
\item
  Track ammunition: \texttt{int\ getAmmo()\ const} and
  \texttt{void\ reload()}
\end{itemize}

\subsection{5. SpaceShip Class}\label{spaceship-class}

A \textbf{SpaceShip} represents the player's ship.

\subsubsection{Attributes:}\label{attributes-2}

\begin{itemize}
\tightlist
\item
  Position (x, y coordinates)
\item
  Velocity vector
\item
  Gun (composition relationship)
\item
  Health points
\item
  Active shots (container of active shots)
\end{itemize}

\subsubsection{Behaviors:}\label{behaviors-2}

\begin{itemize}
\tightlist
\item
  Move the ship based on velocity
\item
  Fire shots using the gun
\item
  Update all active shots
\item
  Remove inactive/out-of-bounds shots
\item
  Set velocity for movement
\end{itemize}

\subsubsection{Requirements:}\label{requirements-2}

\begin{itemize}
\tightlist
\item
  Use \texttt{std::vector} to store active shots
\item
  Implement proper copy/move semantics or delete them
\item
  Provide \texttt{void\ update(double\ deltaTime)} for updating ship and
  shots
\item
  Implement \texttt{void\ setVelocity(const\ Vector2D\&\ vel)}
\item
  Implement \texttt{void\ fireShot()}
\item
  Implement
  \texttt{const\ std::vector\textless{}Shot\textgreater{}\&\ getActiveShots()\ const}
\end{itemize}

\section{Exercises}\label{exercises}

\subsection{Exercise 1: Class Design and
Implementation}\label{exercise-1-class-design-and-implementation}

Design and implement the classes \textbf{Shot}, \textbf{Gun}, and
\textbf{SpaceShip} following modern C++ best practices:

\subsubsection{Part A: Header Files
(.hpp)}\label{part-a-header-files-.hpp}

Create three header files with proper include guards:

\begin{enumerate}
\def\labelenumi{\arabic{enumi}.}
\tightlist
\item
  \textbf{Shot.hpp} - Shot class interface
\item
  \textbf{Gun.hpp} - Gun class interface
\item
  \textbf{SpaceShip.hpp} - SpaceShip class interface
\end{enumerate}

\textbf{Requirements:}

\begin{itemize}
\tightlist
\item
  Use \texttt{\#pragma\ once} or traditional include guards
\item
  Declare all public interfaces
\item
  Keep private members private
\item
  Use forward declarations where possible
\item
  Add documentation comments for public methods
\end{itemize}

\subsubsection{Part B: Implementation Files
(.cpp)}\label{part-b-implementation-files-.cpp}

Create corresponding implementation files:

\begin{enumerate}
\def\labelenumi{\arabic{enumi}.}
\tightlist
\item
  \textbf{Shot.cpp} - Shot class implementation
\item
  \textbf{Gun.cpp} - Gun class implementation\\
\item
  \textbf{SpaceShip.cpp} - SpaceShip class implementation
\end{enumerate}

\textbf{Requirements:}

\begin{itemize}
\tightlist
\item
  Implement all member functions
\item
  Use member initializer lists in constructors
\item
  Implement proper const-correctness
\item
  Handle edge cases (e.g., negative values, null checks)
\end{itemize}

\subsubsection{Part C: Modern C++
Features}\label{part-c-modern-c-features}

Your implementation must use:

\begin{itemize}
\tightlist
\item
  \textbf{Uniform initialization} with \texttt{\{\}}
\item
  \textbf{\texttt{const} correctness} for methods that don't modify the
  object
\item
  \textbf{\texttt{std::optional}} for the Gun's \texttt{fire()} method
  (returns shot or nothing)
\item
  \textbf{\texttt{std::vector}} for managing active shots
\item
  \textbf{RAII principles} for resource management
\item
  \textbf{Default member initialization} where appropriate
\end{itemize}

\subsubsection{Example class skeleton:}\label{example-class-skeleton}

\begin{Shaded}
\begin{Highlighting}[]
\CommentTok{// Shot.hpp}
\PreprocessorTok{\#pragma once}

\KeywordTok{struct}\NormalTok{ Position2D }\OperatorTok{\{}
    \DataTypeTok{double}\NormalTok{ x}\OperatorTok{\{}\FloatTok{0.0}\OperatorTok{\};}
    \DataTypeTok{double}\NormalTok{ y}\OperatorTok{\{}\FloatTok{0.0}\OperatorTok{\};}
\OperatorTok{\};}

\KeywordTok{struct}\NormalTok{ Vector2D }\OperatorTok{\{}
    \DataTypeTok{double}\NormalTok{ x}\OperatorTok{\{}\FloatTok{0.0}\OperatorTok{\};}
    \DataTypeTok{double}\NormalTok{ y}\OperatorTok{\{}\FloatTok{0.0}\OperatorTok{\};}
\OperatorTok{\};}

\KeywordTok{class}\NormalTok{ Shot }\OperatorTok{\{}
\KeywordTok{public}\OperatorTok{:}
\NormalTok{    Shot}\OperatorTok{(}\NormalTok{Position2D pos}\OperatorTok{,}\NormalTok{ Vector2D vel}\OperatorTok{,} \DataTypeTok{double}\NormalTok{ power}\OperatorTok{);}
    
    \DataTypeTok{void}\NormalTok{ update}\OperatorTok{(}\DataTypeTok{double}\NormalTok{ deltaTime}\OperatorTok{);}
    \DataTypeTok{bool}\NormalTok{ isInBounds}\OperatorTok{(}\DataTypeTok{double}\NormalTok{ maxX}\OperatorTok{,} \DataTypeTok{double}\NormalTok{ maxY}\OperatorTok{)} \AttributeTok{const}\OperatorTok{;}
    \DataTypeTok{bool}\NormalTok{ isActive}\OperatorTok{()} \AttributeTok{const}\OperatorTok{;}
    \DataTypeTok{void}\NormalTok{ deactivate}\OperatorTok{();}
    
\NormalTok{    Position2D getPosition}\OperatorTok{()} \AttributeTok{const}\OperatorTok{;}
    \DataTypeTok{double}\NormalTok{ getPower}\OperatorTok{()} \AttributeTok{const}\OperatorTok{;}
    
\KeywordTok{private}\OperatorTok{:}
\NormalTok{    Position2D position}\OperatorTok{;}
\NormalTok{    Vector2D velocity}\OperatorTok{;}
    \DataTypeTok{double}\NormalTok{ destructionPower}\OperatorTok{;}
    \DataTypeTok{bool}\NormalTok{ active}\OperatorTok{\{}\KeywordTok{true}\OperatorTok{\};}
\OperatorTok{\};}
\end{Highlighting}
\end{Shaded}

\subsection{Exercise 2: Simulation
Application}\label{exercise-2-simulation-application}

Create a \textbf{main.cpp} file that demonstrates your classes:

\subsubsection{Requirements:}\label{requirements-3}

\begin{enumerate}
\def\labelenumi{\arabic{enumi}.}
\tightlist
\item
  \textbf{Create a SpaceShip} object at position (100, 100)
\item
  \textbf{Simulate movement} over 10 seconds:

  \begin{itemize}
  \tightlist
  \item
    Update at 60 FPS (deltaTime = 1/60 seconds)
  \item
    Set different velocities for the ship
  \item
    Fire shots at regular intervals
  \end{itemize}
\item
  \textbf{Display information} at each second:

  \begin{itemize}
  \tightlist
  \item
    Ship position
  \item
    Number of active shots
  \item
    Gun ammunition count
  \item
    Gun cooldown status
  \end{itemize}
\item
  \textbf{Remove inactive shots} that go out of bounds
\end{enumerate}

\subsubsection{Example simulation
structure:}\label{example-simulation-structure}

\begin{Shaded}
\begin{Highlighting}[]
\DataTypeTok{int}\NormalTok{ main}\OperatorTok{()} \OperatorTok{\{}
    \CommentTok{// Create spaceship}
\NormalTok{    SpaceShip ship}\OperatorTok{\{\{}\FloatTok{100.0}\OperatorTok{,} \FloatTok{100.0}\OperatorTok{\},}\NormalTok{ Gun}\OperatorTok{\{}\FloatTok{0.5}\OperatorTok{,} \DecValTok{20}\OperatorTok{\}\};}
    
    \CommentTok{// Simulation parameters}
    \KeywordTok{constexpr} \DataTypeTok{double}\NormalTok{ FPS }\OperatorTok{=} \FloatTok{60.0}\OperatorTok{;}
    \KeywordTok{constexpr} \DataTypeTok{double}\NormalTok{ deltaTime }\OperatorTok{=} \FloatTok{1.0} \OperatorTok{/}\NormalTok{ FPS}\OperatorTok{;}
    \KeywordTok{constexpr} \DataTypeTok{double}\NormalTok{ simulationTime }\OperatorTok{=} \FloatTok{10.0}\OperatorTok{;}
    \KeywordTok{constexpr} \DataTypeTok{double}\NormalTok{ worldWidth }\OperatorTok{=} \FloatTok{800.0}\OperatorTok{;}
    \KeywordTok{constexpr} \DataTypeTok{double}\NormalTok{ worldHeight }\OperatorTok{=} \FloatTok{600.0}\OperatorTok{;}
    
    \CommentTok{// Set ship velocity}
\NormalTok{    ship}\OperatorTok{.}\NormalTok{setVelocity}\OperatorTok{(\{}\FloatTok{50.0}\OperatorTok{,} \FloatTok{0.0}\OperatorTok{\});}  \CommentTok{// Move right at 50 units/sec}
    
    \CommentTok{// Simulation loop}
    \DataTypeTok{double}\NormalTok{ elapsedTime }\OperatorTok{=} \FloatTok{0.0}\OperatorTok{;}
    \DataTypeTok{int}\NormalTok{ frameCount }\OperatorTok{=} \DecValTok{0}\OperatorTok{;}
    
    \ControlFlowTok{while} \OperatorTok{(}\NormalTok{elapsedTime }\OperatorTok{\textless{}}\NormalTok{ simulationTime}\OperatorTok{)} \OperatorTok{\{}
        \CommentTok{// Update ship}
\NormalTok{        ship}\OperatorTok{.}\NormalTok{update}\OperatorTok{(}\NormalTok{deltaTime}\OperatorTok{);}
        
        \CommentTok{// Try to fire every 30 frames (0.5 seconds)}
        \ControlFlowTok{if} \OperatorTok{(}\NormalTok{frameCount }\OperatorTok{\%} \DecValTok{30} \OperatorTok{==} \DecValTok{0}\OperatorTok{)} \OperatorTok{\{}
\NormalTok{            ship}\OperatorTok{.}\NormalTok{fireShot}\OperatorTok{();}
        \OperatorTok{\}}
        
        \CommentTok{// Print status every second}
        \ControlFlowTok{if} \OperatorTok{(}\NormalTok{frameCount }\OperatorTok{\%} \DecValTok{60} \OperatorTok{==} \DecValTok{0}\OperatorTok{)} \OperatorTok{\{}
            \BuiltInTok{std::}\NormalTok{cout }\OperatorTok{\textless{}\textless{}} \StringTok{"Time: "} \OperatorTok{\textless{}\textless{}}\NormalTok{ elapsedTime }\OperatorTok{\textless{}\textless{}} \StringTok{"s}\SpecialCharTok{\textbackslash{}n}\StringTok{"}\OperatorTok{;}
            \CommentTok{// Print ship info...}
        \OperatorTok{\}}
        
\NormalTok{        elapsedTime }\OperatorTok{+=}\NormalTok{ deltaTime}\OperatorTok{;}
        \OperatorTok{++}\NormalTok{frameCount}\OperatorTok{;}
    \OperatorTok{\}}
    
    \ControlFlowTok{return} \DecValTok{0}\OperatorTok{;}
\OperatorTok{\}}
\end{Highlighting}
\end{Shaded}

\subsection{Exercise 3: Extended Features
(Optional)}\label{exercise-3-extended-features-optional}

Implement additional features to enhance your design:

\subsubsection{A. Shot Pooling}\label{a.-shot-pooling}

Instead of creating/destroying shots continuously, implement an object
pool:

\begin{itemize}
\tightlist
\item
  Create a fixed pool of Shot objects
\item
  Reuse inactive shots instead of creating new ones
\item
  This improves performance by reducing allocations
\end{itemize}

\subsubsection{B. Different Gun Types}\label{b.-different-gun-types}

Create derived classes for different gun types:

\begin{itemize}
\tightlist
\item
  \textbf{RapidFireGun}: Lower cooldown, lower power
\item
  \textbf{HeavyGun}: Higher cooldown, higher power
\item
  \textbf{BurstGun}: Fires multiple shots at once
\end{itemize}

Use inheritance and virtual functions appropriately.

\subsubsection{C. Ship Upgrade System}\label{c.-ship-upgrade-system}

Add an upgrade system to SpaceShip:

\begin{itemize}
\tightlist
\item
  Health upgrades
\item
  Speed upgrades
\item
  Gun upgrades (swap between different gun types)
\end{itemize}

\subsubsection{D. Statistics Tracking}\label{d.-statistics-tracking}

Add statistics tracking:

\begin{itemize}
\tightlist
\item
  Total shots fired
\item
  Shots currently active
\item
  Maximum shots on screen simultaneously
\item
  Total distance traveled
\end{itemize}

\subsection{Specific Requirements:}\label{specific-requirements}

✓ All classes compile without warnings (\texttt{-Wall\ -Wextra})\\
✓ Proper const-correctness throughout\\
✓ No memory leaks (use smart pointers if dynamic allocation needed)\\
✓ Proper RAII principles\\
✓ Clear separation of .hpp and .cpp files\\
✓ Meaningful variable and function names\\
✓ Simulation produces reasonable output

\section{Tips and Best Practices}\label{tips-and-best-practices}

\subsection{Design Tips:}\label{design-tips}

\begin{enumerate}
\def\labelenumi{\arabic{enumi}.}
\tightlist
\item
  \textbf{Start simple}: Get basic functionality working before adding
  features
\item
  \textbf{Think about ownership}: Who owns the shots? The gun or the
  ship?
\item
  \textbf{Consider the update loop}: How do objects update each frame?
\item
  \textbf{Bounds checking}: When should shots be deactivated?
\end{enumerate}

\subsection{C++ Best Practices:}\label{c-best-practices}

\begin{Shaded}
\begin{Highlighting}[]
\CommentTok{// Good: Use const references for parameters}
\DataTypeTok{void}\NormalTok{ setVelocity}\OperatorTok{(}\AttributeTok{const}\NormalTok{ Vector2D}\OperatorTok{\&}\NormalTok{ vel}\OperatorTok{);}

\CommentTok{// Good: Mark non{-}modifying functions const}
\NormalTok{Position2D getPosition}\OperatorTok{()} \AttributeTok{const}\OperatorTok{;}

\CommentTok{// Good: Use member initializer lists}
\NormalTok{Shot}\OperatorTok{::}\NormalTok{Shot}\OperatorTok{(}\NormalTok{Position2D pos}\OperatorTok{,}\NormalTok{ Vector2D vel}\OperatorTok{,} \DataTypeTok{double}\NormalTok{ power}\OperatorTok{)}
    \OperatorTok{:}\NormalTok{ position}\OperatorTok{\{}\NormalTok{pos}\OperatorTok{\},}\NormalTok{ velocity}\OperatorTok{\{}\NormalTok{vel}\OperatorTok{\},}\NormalTok{ destructionPower}\OperatorTok{\{}\NormalTok{power}\OperatorTok{\},}\NormalTok{ active}\OperatorTok{\{}\KeywordTok{true}\OperatorTok{\}} \OperatorTok{\{\}}

\CommentTok{// Good: Use std::optional for operations that may fail}
\BuiltInTok{std::}\NormalTok{optional}\OperatorTok{\textless{}}\NormalTok{Shot}\OperatorTok{\textgreater{}}\NormalTok{ Gun}\OperatorTok{::}\NormalTok{fire}\OperatorTok{()} \OperatorTok{\{}
    \ControlFlowTok{if} \OperatorTok{(!}\NormalTok{canFire}\OperatorTok{())} \OperatorTok{\{}
        \ControlFlowTok{return} \BuiltInTok{std::}\NormalTok{nullopt}\OperatorTok{;}
    \OperatorTok{\}}
    \CommentTok{// ... create and return shot}
\OperatorTok{\}}

\CommentTok{// Good: Use range{-}based for loops}
\ControlFlowTok{for} \OperatorTok{(}\KeywordTok{auto}\OperatorTok{\&}\NormalTok{ shot }\OperatorTok{:}\NormalTok{ activeShots}\OperatorTok{)} \OperatorTok{\{}
\NormalTok{    shot}\OperatorTok{.}\NormalTok{update}\OperatorTok{(}\NormalTok{deltaTime}\OperatorTok{);}
\OperatorTok{\}}
\end{Highlighting}
\end{Shaded}

\subsection{Common Pitfalls to Avoid:}\label{common-pitfalls-to-avoid}

❌ Forgetting to update cooldown timers\\
❌ Not removing out-of-bounds shots (memory leak)\\
❌ Mixing up position and velocity\\
❌ Not handling edge cases (negative values, division by zero)\\
❌ Using raw \texttt{new}/\texttt{delete} instead of smart pointers or
containers\\
❌ Making everything public

\section{Additional Resources}\label{additional-resources}

\begin{itemize}
\tightlist
\item
  \textbf{C++ Reference}:
  \href{https://cppreference.com}{cppreference.com}
\item
  \textbf{C++ Core Guidelines}:
  \href{https://isocpp.github.io/CppCoreGuidelines}{isocpp.github.io/CppCoreGuidelines}
\item
  \textbf{std::optional}:
  \href{https://en.cppreference.com/w/cpp/utility/optional}{en.cppreference.com/w/cpp/utility/optional}
\item
  \textbf{RAII}:
  \href{https://en.cppreference.com/w/cpp/language/raii}{en.cppreference.com/w/cpp/language/raii}
\end{itemize}

\begin{center}\rule{0.5\linewidth}{0.5pt}\end{center}

\textbf{Good luck and happy coding!} 🚀

\emph{Remember: Write code that you would want to maintain in 6 months.
Your future self will thank you!}




\end{document}
